\documentclass{beamer}
\usetheme{Boadilla}
\usepackage{amsmath}
\usepackage{empheq}
\usepackage{appendixnumberbeamer}
\graphicspath{{./figures/}}

%--- ADD AUTOMATIC TITLE TO FRAMES ---%
\addtobeamertemplate{frametitle}{
   \let\insertframetitle\insertsectionhead}{}
\addtobeamertemplate{frametitle}{
   \let\insertframesubtitle\insertsubsectionhead}{}
\makeatletter
  \CheckCommand*\beamer@checkframetitle{\@ifnextchar\bgroup\beamer@inlineframetitle{}}
  \renewcommand*\beamer@checkframetitle{\global\let\beamer@frametitle\relax\@ifnextchar\bgroup\beamer@inlineframetitle{}}
\makeatother
%-------------------------------------%


\begin{document}

\title[Syntrophic MCs]{The impact of syntrophic interaction on microbial communities}
\author[LB \& APG]{L\'eo Buchenel \& Alberto Pascual-Garc\'ia}
\date{\today}
\institute{ETH Z\"urich}

\thispagestyle{empty}
\begin{frame}
\titlepage
\end{frame}

\section*{Presentation plan}
\begin{frame}
\tableofcontents
\end{frame}

\section{Introduction to the model}
\subsection{Syntrophic interaction}
\begin{frame}
\textbf{INSERT DRAWING OF SYNTROPHIC MCs}

\end{frame}

\subsection{Processes at play}
\begin{frame}
\textbf{INSERT EXPLANATION OF ALL DIFFERENT PROCESSES AT PLAY (DRAWING)}
\end{frame}

\subsection{Dynamical Equations}
\begin{frame}
Temporal evolution of the resources $R_\mu$ ($\mu = 1, \dots, N_R$) and the consumers $S_i$ ($i=1, \dots, N_S$):
\begin{subequations}
\begin{empheq}[left=\empheqlbrace]{align}
\dot{R}_\mu= &  \left(\sum_{j=1}^{N_S} \gamma_{j \mu} S_j -m_\mu \right)R_\mu + \sum_{j=1}^{N_S} \alpha_{\mu j} S_j + l_\mu   \\
\dot{S}_i = & \left(\sum_{\nu=1}^{N_R} \sigma_{i\nu} \gamma_{i\nu} R_\nu - d_i - \sum_{\nu=1}^{N_R} \alpha_{\nu i}\right)S_i
\end{empheq}
\end{subequations}
\end{frame}

\subsection{Metaparameters framework}
\begin{frame}
\textbf{INSERT DIAGRAM WHERE EXPLAIN METAPARAMETERS}
\end{frame}

\section{Feasibility}
\subsection{Physical Requirements}
\begin{frame}
Impose two conditions:
\begin{itemize}
\item Conservation of biomass
\item Positivity of the parameters
\end{itemize}
$\rightarrow$ restrictions on parameters $\rightarrow$ restrictions on metaparameters
\end{frame}
\subsection{Feasibility profile}
\begin{frame}
Without syntrophy: full feasibility for $\gamma_0$ below curve $ \sim S_0$
\vfill
\begin{center}
\includegraphics[width=0.75\linewidth]{{typical_feasibility_volume}.pdf}
\end{center}
\end{frame}

\subsection{Fully feasible region}
\begin{frame}
%% SAY THAT CAN FOCUS ON FULLY FEASIBLE REGION BECAUSE OF THIS
Addition of syntrophy $\rightarrow$ only high $\gamma_0$, low $S_0$ remain feasible
\vfill
\begin{center}
\includegraphics[width=0.75\linewidth]{{feasibility_region_wt_NR25_NS25_Nest0.15_Conn0.1808}.pdf}
\end{center}
\end{frame}

\section{Dynamical stability}
\subsection{The Jacobian}
\begin{frame}
\begin{itemize}
\item Jacobian of the system determines dynamical stability of 
\end{itemize}
\end{frame}

\section{Structural stability}
\begin{frame}
\end{frame}

\appendix

\section{Backup slides -- Feasibility}
\subsection{Physical Requirements}
\begin{frame}
\begin{itemize}
\item We impose that -- at equilibrium -- no biomass can be created out of nothing, which is translated mathematically by the constraint:
\begin{equation}
\sum_{\nu = 1}^{N_R} \left(1-\sigma_{i\nu}\right)\gamma_{i\nu}R^*_\nu \geq \sum_{\nu=1}^{N_R} \alpha_{\nu i} \quad \forall i = 1, \dots, N_S. \label{eq : conservation of biomass}
\end{equation}

\item We need every parameter in the model to be positive. In total there are $3 N_R + 2 N_S + 3 N_R N_S$ parameters, constrained by the $N_R + N_S$ fixed points equations. So we pick $2 N_R + N_S + 3 N_R N_S$ positive parameters and we make sure they verify:
\begin{subequations}\label{eq : positive parameters}
\begin{empheq}[left=\empheqlbrace]{align}
d_i &= \sum_{\nu = 1}^{N_R} \left(\sigma_{i\nu} \gamma_{i\nu} R^*_\nu -\alpha_{\nu i}\right) > 0 \quad \forall i =1, \dots, N_S, \\
m_\mu &= \frac{l_\mu - \sum_{j=1}^{N_S} \left(\gamma_{j\mu} R^*_\mu -\alpha_{\mu j}\right) S_j^*}{R^*_\mu} > 0 \quad \forall \mu = 1, \dots, N_R.
\end{empheq}
\end{subequations}

\end{itemize}
\end{frame}

\subsection{Theoretical predictions}
\begin{frame}
\begin{itemize}
\item \textbf{Fully feasible region:} Transforming Eqs.\eqref{eq : conservation of biomass} and \eqref{eq : positive parameters} in their metaparameters form, we get an equation that roughly  describes the fully feasible zone:
\begin{subequations}\label{eq : fully feasible volume}
\begin{empheq}{align}
 \min(1-\sigma_0, \sigma_0) \gamma_0 R_0 \gtrapprox \max_i\left\{\frac{\deg(A,i)}{\deg(G,i)}\right\} \alpha_0\\
\gamma_0 R_0 \lessapprox  \min_\nu \left\{ \frac{l_0}{\deg(G,\nu) S_0} + \frac{\deg(A,\nu)}{\deg(G,\nu)}\alpha_0\right\}
\end{empheq}
\end{subequations}

\end{itemize}
\end{frame}

\end{document}
