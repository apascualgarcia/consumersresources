\documentclass[12pt, titlepage]{report}
\usepackage{consumer_resource_final}
\graphicspath{{./figures/}}

\begin{document}
Very similarly to dynamical stability, where the abundances of the resources and species are changed, one can perturb the parameters of a model at an equilibrium point. Namely the idea of \textit{structural stability} is the following : one takes a given equilibrium point [\textbf{Check whether it needs to be linearly stable or not}] $\{R^*, S^*\}$ and changes some or all the parameters of the model. We will focus here on perturbing the external feeding rate of the resources\footnote{This choice may seem a bit arbitrary at first hand. It stems from the idea that we wanted to keep things simple and hence decided to change only one parameter of the model. We chose the external feeding rate of the resources because it simply is the easiest one to control in an actual experimental setup.} :
\begin{equation}
l_\mu \rightarrow \tilde{l}_\mu \equiv l_\mu \left(1+\Delta_S\right).
\end{equation}
The system is then time evolved under the equations of evolution \eqref{eq : differential eq for resources and species}, except that $l_\mu \rightarrow \tilde{l}_\mu$, until an equilibrium $\{\tilde{R}^{*}, \tilde{S}^{*}\}$ is reached. The same metrics as before can be used to quantify the structural stability.

\end{document}
