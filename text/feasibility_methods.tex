\documentclass[12pt, titlepage]{report}
\usepackage{consumer_resource_final}
\graphicspath{{./figures/}}

\begin{document}


% We want to be able to build feasible models numerically, \ie we would like to generate a set of constant numbers $\{
% l_\nu, m_\nu, R^*_\nu, S^*_j, \gamma_{j\nu}, \alpha_{\nu j}, \sigma_{j\nu}\}$ such that the equilibria equations Eqs.\eqref{eq : equilibrium resources and species} are fulfilled.
Since its very inception \cite{may_will_1972}, the study of ecological interactions has been and still is tightly close to the one of random matrices \cite{allesina_stability_2012, allesina_predicting_2015, barbier_cavity_2017}. Usually, the procedure is we assume a feasible equilibrium point, where some matrix of the model (\eg the species-interaction matrix or the jacobian) is approximated as random, and then study the dynamical or structural stability of said feasible point.

That framework is not satisfying for the study we would like to conduct, because it does not take time to study whether random parameters make sense in the first place.
Indeed, before studying whether a microbial community can sustain perturbations, we need to know if said community actually \important{exists}. Biological systems, like any other natural systems, are constrained by laws, whether they arise from physical or biological considerations. For instance, it would not make sense to consider microbial communities that \eg violate the laws of thermodynamics. In the following section, we explain how such considerations can help determining the answer to the \important{feasibility} question:

\begin{centering}
\fbox{\begin{minipage}{\linewidth}
\itshape
Can microbial communities arising from a random set of parameters make sense on a physical and biological level?
If not, what are the conditions that should be imposed and how are these translated mathematically?
\end{minipage}}
\end{centering}
\subsection{Basic concepts}\label{sec : methods feasibility basic concepts}
As explained above, we want to impose conditions such that we only study systems that are compatible with biological and physical laws. Choosing such restrictions is a crucial task : we want to be as close to nature as possible but we also need to stay simple enough such that the model remains mathematically tractable. Our %-- somewhat arbitrary but still very justified ---
choice is the following :
 % Indeed for our model to make sense,
  any system deemed as feasible must have ``biological'' model parameters and conserve biomass.

Asking for the model parameters to be ``biological'' means we want them to carry their intended biological interpretation. This means \eg that any syntrophic interaction has to be non-negative $\alpha_{\mu i} \geq 0 $ otherwise it cannot be interpreted as a syntrophic interaction anymore! More generally, the values of the parameters will be restricted.
%this is equivalent to requiring that all the model parameters are non-negative:
% \begin{equation}
%  p \geq 0 \ \forall p \in \mathcal{P}.
% \end{equation}
% In our study, this equation will be slightly restricted since
Namely, we are looking for positive-valued equilibria. %, so we require $R^*_\mu, S^*_i > 0$ specifically for these two parameters.
Also, we require that every consumer can allocate some of each resource it consumes to growth\footnote{It wouldn't make sense to say that species $i$ eats resource $\mu$ with efficiency $0$, since this is equivalent to species $i$ not eating resource $\mu$, and this is already encoded in the network structure.} : zero efficiencies are forbidden. Finally every resource external feeding rate should be non-zero in order to avoid resource depletion and every resource and consumer must eventually die out in the absence of interaction. Mathematically, these considerations are equivalent to:
\begin{equation}
\boxed{
R^*_\mu, S^*_i, \sigma_{i\mu}, l_\mu, d_i, m_\mu, \sigma_{i\mu} > 0 \text { and } \gamma_{i\mu}, \alpha_{\mu i} \geq 0.
}
\label{eq : feasibility positive parameters}
\end{equation}

That condition already greatly restricts the choice of parameters $p\in \mathcal{P}$. However, additional complexity arises from the relationships parameters have to follow by definition. Indeed, the $3 N_R +2 N_S + 3 N_R N_S $ parameters are constrained by the $N_R + N_S $ equations \eqref{eq: equilibrium resources and species}. So if we choose $2 N_R + N_S + 3 N_R N_S$ parameters, the remaining $N_R + N_S$ are instantly determined. Traditionally, we would solve for $R^*$ and $S^*$ and choose the rest of the parameters, but for reasons explained in Appendix \ref{sec: explanation solve for d and m}, we will solve for the consumers death rates $d_i$ and the resources diffusion rate $m_\mu$. This means that if we \important{choose} non-negative $\gamma, \alpha, \sigma, \tau, l, R^*$ and $S^*$, Eqs.\eqref{eq : feasibility positive parameters} can be combined with Eqs.\ref{eq: equilibrium resources and species} into:
\begin{subequations}\label{eq : feasibility positive d and m}
\begin{empheq}[left=\empheqlbrace]{align}
d_i &= \sum_\nu \left( \sigma_{i\nu} \gamma_{i\nu} R^*_\nu - \alpha_{\nu i} \right) > 0 \ \forall i=1, \dots, N_S \label{eq : feasibility positive d}\\
m_\mu &= \frac{l_\mu - \sum_j \left(\gamma_{j\mu}R^*_\mu-\alpha_{\mu j}\right)S^*_j}{R^*_\mu} > 0 \ \forall \mu = 1, \dots, N_R \label{eq : feasibility positive m}
\end{empheq}
\end{subequations}

In addition to these constraints, any feasible system should conserve biomass \important{at equilibrium}\footnote{This weak condition should hold only at equilibrium : we allow transition periods where biomass may not be conserved.}: no species should be able to produce more biomass than it physically can. More specifically, a consumer $i$ attains, from resources consumption, a total biomass of $\sum_\nu \gamma_{i\nu}R^*_\nu S^*_i$.
From this available biomass, only $\sum_\nu \sigma_{i\nu}\gamma_{i\nu}R^*_\nu S^*_i$ is devoted to growth. Out of the remaining $\sum_\nu (1-\sigma_{i\nu})\gamma_{i\nu}R^*_\nu S^*_i$, a part $\sum_\nu \alpha_{\nu i} S^*_i$ is given back to the resources as a syntrophic interaction. We simply impose that the syntrophic interaction is smaller than or equal to the available remaining biomass :
\begin{equation}\label{eq : feasability biomass conservation}
 \sum_\nu (1-\sigma_{i\nu})\gamma_{i\nu}R^*_\nu  \geq \sum_\nu \alpha_{\nu i} \ \forall i=1, \dots, N_S.
\end{equation}
From now on, we will say that \textbf{a parameter set $p$ is \define{feasible} if it satisfies Eqs.\eqref{eq : feasibility positive d and m} and \eqref{eq : feasability biomass conservation}}.
This is completely deterministic, in the sense that for a given parameters set $p \in \mathcal{P}$ one can without a doubt say whether it is feasible or not.
Hence we define the \define{parameters set feasibility function} $\mathfrak{F} : P \rightarrow \{ 0, 1 \}$, which takes a parameter set as an input and tells you whether this parameter set is feasible or not:
\begin{equation}
\mathfrak{F}(p)=
\begin{cases}
1 \text{ if }p \text{ is feasible,} \\
0 \text{ else.}
\end{cases}
\end{equation}
However as explained in the introduction we will usually not work with a parameter set $p \in \mathcal{P}$ directly -- because there are too many variables to keep track of -- but with a metaparameter set $m \in \mathcal{M}$ and a consumption-syntrophy network $(G,A) \in \mathcal{B}_{N_S \times N_R} \times \mathcal{B}_{N_R \times N_S}$ instead. We can  define a corresponding
\define{metaparameters set feasibility function} $\mathcal{F} : \mathcal{M} \rightarrow [0, 1] \times \mathcal{B}_{N_R \times N_S}$ which is the probability that a given set of metaparameters $m \in \mathcal{M}$ coupled with binary matrices $B=(G, A)$ gives rise -- through the algorithmic procedure $\mathcal{A}$ -- to a feasible parameter set :
\begin{equation}\boxed{
\mathcal{F}(m, B)=\text{Probability}\left\{\mathfrak{F}\left(\mathcal{A}(m, B)\right)=1\right\} \label{eq : feasibility methods feasibility metaparameters function}
}
\end{equation}
%We will in general work with $\mathcal{F}$ rather than $\mathfrak{F}$ because it is easier to handle metaparameters.
In practice $\mathcal{F}(m, B)$ is estimated numerically by generating $N$ parameters sets from $(m,B)$ and calculating the number of feasible ones :
\begin{equation}
\mathcal{F}(m, B) = \lim_{N\rightarrow \infty} \sum_{i=1}^N \frac{\mathfrak{F}(\mathcal{A}(m,B))}{N} \approx \sum_{i=1}^N \frac{\mathfrak{F}(\mathcal{A}(m,B))}{N} \text{ for } N \gg 1.
\end{equation}

\subsection{The feasibility region}\label{sec : methods feasibility volume}
Appendix \ref{sec : feasibility methods algorithmic procedure} explains the algorithmic procedure $\mathcal{A}(m, B) \in \mathcal{P}$ which allows us to build feasible parameters out of a set of metaparameters and a consumption-syntrophy network. However, in order to work properly, the combination of metaparameters used as an input of the algorithm must most of the time lead to the realisation of feasible systems. We hence need to find what region of the metaparameters space lead to a high feasibility : this is precisely the idea behind the notion of the feasibility region discussed below.

But first, let's see how our study can made simpler. Feasibility conditions discussed above tell us that we may choose six metaparameters\footnote{Indeed, we saw that $d_i$ and $m_\mu$ are set by the other parameters, so we cannot freely choose $d_0$ and $m_0$.} : $\gamma_0$, $\alpha_0$, $l_0$, $\sigma_0$, $S_0$ and $R_0$. However, following the analysis of \cite{barbier_cavity_2017}, we notice that our model Eqs.\eqref{eq: differential eq for resources and species} still possesses some freedom. Indeed we can choose the set of units we work in to fix the values of some metaparameters. There are two physical quantities at stake here : biomass and time, and we may choose, however we want it, a specific set of units describing both of them.
We measure biomass in units of the average resource abundance at equilibrium\footnote{That choice is not completely innocent. Indeed we will see later that the matrix $\alpha_{\nu i}-\gamma_{i \nu} R^*_\nu$ is a crucial quantity here. Setting $\av{R^*}=1$ allows us to simply study the impact of $\gamma$ against $\alpha$ instead of the more complicated $\gamma R^*$ versus $\alpha$.}:
\begin{equation}
 \av{R_\mu} = R_0 = 1.
\end{equation}
Similarly, we measure time such that the average external resource uptake rate is one, that is:
\begin{equation}
\av{l_\mu} = l_0 = 1.
\end{equation}
After this manipulation, the number of metaparameters is reduced from six to four : only $\gamma_0$, $S_0$, $\alpha_0$ and $\sigma_0$ remain.

For the sake of simplicity, we keep the same $\sigma_0$ throughout our whole study. We take a value close to the efficiency of real microbial systems [\textbf{insert ref}], that is $\sigma_0 =0.25$.

Overall, we need to choose the last three remaining metaparameters: $\alpha_0$, $\gamma_0$ and $S_0$. We will modify these metaparameters throughout the study. Since the remaining eight are fixed, we sometimes will elude them in the notation and will write instead of $m=(\gamma_0, S_0, \alpha_0, \sigma_0, R_0, l_0, d_0, m_0) \in \mathcal{M}$ simply $m=(\gamma_0, S_0, \alpha_0)$.
%As soon as $\gamma_0$ and $S_0$ are chosen, we get a range of $\alpha_0$ which gives rise to feasible systems.
%We will then choose $\gamma_0$ and $S_0$ such that they lead to feasible systems for every consumption matrix considered here \textbf{when there is no syntrophy}, \ie $\alpha_0=0$. We will then study the impact of varying $\alpha_0$ at those values of $\gamma_0$ and $S_0$.

Formally, we can define for a consumption matrix $G$ coupled with a syntrophy adjacency matrix $A$ the $x$-feasible volume $\mathcal{F}^{G,A}_x$ of the metaparameters space $\mathcal{M}$ that will lead to at least a ratio $x$ of feasible systems \ie :
\begin{equation}
\mathcal{F}^{G,A}_x \defined \left\{m \in \mathcal{M} : \mathcal{F}\left(m, (G,A)\right)\geq x \right\}. \label{eq : x feasible volume}
\end{equation}
It is clear that $\mathcal{F}^{G,A}_0 = \mathcal{M}$ $\forall G$ and $\text{Vol}\left(\mathcal{F}^{G,A}_{x}\right) \leq \text{Vol}\left(\mathcal{F}^{G,A}_{y}\right)$ $\forall x > y, G$. We can similarly define for a set $S = \left\{ (G_1,A_1) , (G_2, A_2), \dots, (G_N, A_N)\right\}$ of $N$ couples of matrices their \important{common feasibility} region $\mathcal{F}^S_x$, which is the region of the metaparameters space where feasibility is at least $x$ for every couple in the set:
\begin{equation}
\mathcal{F}^S_x \defined \intersection{(G,A) \in S} \mathcal{F}^{G,A}_x.
\end{equation}
We also define for a matrix set $S$, its critical feasibility $f^*(S)$, which is the largest feasibility we can get while still having a non-zero common volume :
\begin{equation}
f^*(S) \defined \max_{x \in \left[0,1\right]}\left\{ x : \text{Vol}\left(\mathcal{F}^S_x\right) > 0 \right\}. \label{eq : feasibility methods critical common feasibility}
\end{equation}
For actual computations, we will choose a matrix set $S_M$, stick to it during the whole thesis, and work in its critical feasibility region $\mathcal{F}^*$, defined as :
\begin{equation}\boxed{
\mathcal{F}^* \defined \mathcal{F}^{S_M}_{f^*(S_M)}. \label{eq : feasibility methods critical common feasibility region}
}
\end{equation}
Our hope is that we may find a fully feasible common region, \ie $f^*(S)=1$. 

\subsection{Estimating the fully feasible region $\mathcal{F}^{G,A}_1$}
Now that we defined the $x$-feasible volume of a given couple consumption-syntrophy network $(G,A)$ in Eq.\eqref{eq : x feasible volume}, we would like to know what regions of the metaparameters space lead to fully feasible systems. We imposed two conditions that characterise the set of feasible parameters: Eqs.\eqref{eq : feasibility positive d and m} and \eqref{eq : feasability biomass conservation}. We use them as a start to get corresponding metaparameters equations that describe $\mathcal{F}^{G,A}_1$.
\subsubsection{Biomass conservation}
As stated above, we require that biomass is conserved in our model. This is equivalent to fulfilling Eq.\eqref{eq : feasability biomass conservation}, which we rewrite here:
\begin{equation}
\sum_\nu \left(1-\sigma_{i\nu}\right)\gamma_{i\nu}R^*_\nu \geq \sum_\nu \alpha_{\nu i} \ \forall i=1,\dots, N_S.
\end{equation}
% The idea is to neglect the variance of every quantity involved \ie we use the approximation
% \begin{equation}
% \gamma_{i\mu} \approx \gamma_0 G_{i\mu}, \ \alpha_{\mu i}\approx \alpha_0 A_{\mu i}, \ \sigma_{i\nu} = \sigma_0 \text{ and }R^*_\nu \approx R_0.
% \end{equation}
%This means the RHS of Eq.\eqref{eq : feasability biomass conservation} is roughly given by
Eqs.\eqref{eq: metaparameters approximations} can be used to estimate the RHS of this equation:
\begin{equation}
\sum_\nu \alpha_{\nu i} \approx \deg(A, i) \alpha_0,
\end{equation}
where $\deg(A,i)$ is the degree of the $i$-th column of the $\alpha$ matrix :
\begin{equation}
\deg(A, i) = \sum_\nu A_{\nu i}.
\end{equation}
Similarly,
\begin{equation}
\sum_\nu \left(1-\sigma_{i\nu}\right)\gamma_{i\nu} R^*_\nu \approx (1-\sigma_0)R_0\sum_{\nu}\gamma_{i\nu} \approx \deg(G, i)(1-\sigma_0)R_0\gamma_0,
\end{equation}
Energy conservation Eq.\eqref{eq : feasability biomass conservation} is then equivalent to
\begin{equation}
\deg(A,i) \alpha_0 \lessapprox \deg(G,i) (1-\sigma_0)R_0\gamma_0 \ \forall i=1,...,N_S
\end{equation}
Since $\deg(G,i) > 0 $, we have\footnote{Indeed, $\deg(G,i)$ is the number of resources species $i$ eats. We of course ask every consumer to at least consume something, otherwise they would not be part of the microbial community.}:
\begin{equation}
\frac{\deg(A,i)}{\deg(G,i)} \alpha_0 \lessapprox (1-\sigma_0)R_0\gamma_0 \ \forall i=1,...,N_S
\end{equation}
This is fulfilled if :
\begin{equation}\label{eq: feasability energy conservation}
\boxed{
\max_i\left\{\frac{\deg(A,i)}{\deg(G,i)}\right\} \alpha_0 \lessapprox (1-\sigma_0)R_0 \gamma_0
}.
\end{equation}
  Systems where the ratio $\frac{\# \text{resources released to}}{\# \text{resources consumed}}$ is small for each species allow for a larger individual syntrophy interaction (which is very intuitive).
\subsubsection{Biological interpretation of the parameters}
Additionally, the consumers death rates $d_i$ have to be positive. This implied Eq.\eqref{eq : feasibility positive d}, which may be recast as :
\begin{equation}
\sum_\mu \sigma_{i\mu}\gamma_{i\mu}R^*_\mu > \sum_\mu \alpha_{\mu i}
\end{equation}
Using a reasoning similar to above, we get a corresponding metaparameters inequality:
\begin{equation} \label{eq : feasability positivity d}
\boxed{
\max_i\left\{\frac{\deg(A,i)}{\deg(G,i)}\right\} \alpha_0 \lessapprox \sigma_0R_0 \gamma_0
}.
\end{equation}
Also, the resources diffusion rates $m_\nu$ need to be positive:
\begin{equation}
l_\nu + \sum_j \alpha_{\nu j} S^*_j > \sum_j \gamma_{j\nu}R^*_\nu S^*_j \ \forall \nu=1,\dots,N_R
\end{equation}
Which is equivalent to
\begin{equation}
l_0 + \deg(A, \nu) \alpha_0 S_0 \gtrapprox \deg(G,\nu) \gamma_0 R_0 S_0 \ \forall \nu
\end{equation}
Since $\deg(G,\nu)>0$, we\footnote{Similarly to a previous footnote, we require that every resource $\nu$ is eaten by at least one consumer, \ie $\deg(G,\nu)>0$, otherwise it does not belong to the community.} can divide the above equations by $\deg(G,\nu)>0$ and then recast these $N_R$ equations into a single condition:
\begin{equation} \label{eq : feasability positivity m}
\boxed{
\min_\nu\left\{\frac{l_0}{\deg(G,\nu) S_0} + \frac{\deg(A,\nu)}{\deg(G,\nu)}\alpha_0\right\} \gtrapprox \gamma_0 R_0
}
\end{equation}
This says that systems where the ratio $\frac{\#\text{number of species that release to me}}{\#\text{number of species that consume me}}$ is large for every resource are more feasible. The strategy should be then to have $\gamma$'s that have large $\deg(G,\nu)$ (\ie resources are consumed by many species) and large $\deg(G,i)$ (\ie species consume a lot of species), and the other way around for $\alpha$ (not sure about this for the last one).

\subsubsection{Combining both conditions}
The two upper bounds Eqs.\eqref{eq: feasability energy conservation}-\eqref{eq : feasability positivity d} on $\alpha_0$ can be combined in a single inequality :
\begin{equation} \label{eq : largest feasible alpha0 with structure}
\max_i\left\{\frac{\deg(A,i)}{\deg(G,i)}\right\} \alpha_0 \lessapprox \min(1-\sigma_0, \sigma_0) \gamma_0 R_0
\end{equation}
Note that when $\alpha_0 > 0$, we will trivially require that the syntrophy matrix is not empty, \ie there exists at least an $i$ for which $\deg(A,i) \geq 1$. Also, the largest value $\deg(G,i)$ can get (for any $i$) is $N_R$. Hence,
\begin{equation}
\max_i\left\{ \frac{\deg(A,i)}{\deg(G,i)} \right\} \geq \frac{1}{N_R},
\end{equation}
and we can find the largest allowed theoretical non-zero $\alpha_0$ :
\begin{equation}
\boxed{
\alpha_0 \lessapprox \min(1-\sigma_0, \sigma_0) \gamma_0 R_0 N_R. \label{eq : largest feasible alpha0}
}
\end{equation}
Finally, Eq.\eqref{eq : feasability positivity m} and \eqref{eq : largest feasible alpha0 with structure} can be combined into a single one, which characterises the fully feasible region $\mathcal{F}^{G,A}_1$:
\begin{equation}
\boxed{
\max_i\left\{\frac{\deg(A,i)}{\deg(G,i)}\right\} \alpha_0
\lessapprox \min(1-\sigma_0, \sigma_0) \gamma_0 R_0
\lessapprox
\min \left(1-\sigma_0, \sigma_0 \right) \min_\nu \left\{ \frac{l_0}{\deg(G,\nu) S_0} + \frac{\deg(A,\nu)}{\deg(G,\nu)}\alpha_0\right\}
}\label{eq : fully feasible volume}
\end{equation}


% \paragraph{Energy conservation/dissipation}
% The first condition we will impose on our systems is that they do not create new matter.
%
% Remember that in our model, the total amount of biomass given to the species $i$ by the resources is $\sum_\nu \gamma_{i\nu}R_\nu S_i$.
% However species $i$ will not allocate all of this biomass to growth. As seen in Eq.\eqref{eq : differential eq for species}, only a fraction $\sigma_{i\nu}$ will be used in this purpose. This means species $i$ disposes of $\sum_\nu(1-\sigma_{i\mu})\gamma_{i\mu}R_\mu S_i$ biomass to complete other processes.
% We know that one of these is producing byproducts (\ie the syntrophic interaction) at a total rate $\sum_\nu \alpha_{\nu i} S_i$. Because this biomass is produced in the cell, it has to come from the biomass the cell disposes of, which naturally leads to the condition\footnote{Note that the condition is a bit relaxed here. Biomass cannot be created at equilibrium. However we allow some transient regimes where this momentarily can occur, \eg after a big shock inflicted to the system.}:
%   \begin{equation}
%   \sum_{\nu} \left(1-\sigma_{i\nu}\right)\gamma_{i\nu}R^*_\nu \geq \sum_\nu \alpha_{\nu i} \label{eq : dissipation constraint}
% \end{equation}
% The above equation will be referred to as the \textit{conservation of biomass constraint}.
%
% The idea is to find metaparameters such that this constraint is automatically satisfied (which eases building the system numerically). This is easily done by finding the minimum of the LHS and maximum of RHS of Eq.\eqref{eq : dissipation constraint}. Indeed :
% \begin{equation}
%   \sum_{\nu} \left(1-\sigma_{i\nu}\right)\gamma_{i\nu}R^*_\nu \geq (1-\maxd{\sigma})\mind{\gamma}\mind{R^*},
% \end{equation}
% where \ $\mind{ }$ \ denotes the minimum value of the random variable and \ $\maxd{ }$ \ its maximum value. On the other hand,
% \begin{equation}
%   \sum_\nu \alpha_{\nu i} \leq \maxd{\alpha} N_R.
% \end{equation}
% This means that if we take metaparameters such that
% \begin{equation}
%   \maxd{\alpha} N_R < (1-\maxd{\sigma})\mind{\gamma}\mind{R^*}, \label{eq : min and max energy constraint}
% \end{equation}
% then Eq.\eqref{eq : dissipation constraint} is automatically followed.
%
% Because of the way we choose our variables we have for every random variable in the problem,
% \begin{equation}
% \mind{X} = (1-\epsilon)\mean{X} \text{ and }\maxd{X} = (1+\epsilon)\mean{X}
% \end{equation}
% where \ $\mean{X}$ \ denotes the mean of $X$. This means Eq.\eqref{eq : min and max energy constraint} is equivalent to, in terms of metaparameters:
% \begin{equation}
%   \alpha_0 < \frac{\left(1-\epsilon\right)^2}{1+\epsilon}\left(1-\left(1+\epsilon\right)\sigma_0\right)\frac{\gamma_0 R_0}{N_R}.
% \end{equation}
% In the $\epsilon \ll 1$ limit, this is equivalent to:
% \begin{equation}
%   \alpha_0 < \left(1-3\epsilon\right)\left(1-(1+\epsilon)\sigma_0\right)\frac{\gamma_0 R_0}{N_R}.
% \end{equation}
% % \begin{equation}
% %   \sum_{\nu} \left(1-\sigma_{i\nu}\right)\gamma_{i\nu}R^*_\nu \geq \min_\nu\left(\left(1-\sigma_{i\nu}\right)R^*_\nu\right) \sum_\nu \gamma_{i\nu} \geq \left(1-\max_\nu\left(\sigma_{i\nu}\right)\right)\min_\nu\left(R^*_\nu\right)N_R \gamma_0.
% % \end{equation}
% % Now, the way we draw the $\sigma_{i\nu}$ numerically is effectively by drawing elements of a uniform distribution of mean $\sigma_0$, i.e.
% % \begin{equation}
% %   \sigma_{i\nu} \approx \text{Unif}(0, 2\sigma_0).
% % \end{equation}
% % It is quite easy to estimate $\max_\nu(\sigma_{i\nu})$, indeed we can easily show that :
% % \begin{equation}
% %   E\left[\max_\nu\left(\sigma_{i\nu}\right)\right] = 2 \sigma_0 \frac{N_R}{N_R+1}.
% % \end{equation}
% % This has a standard deviation
% % \begin{equation}
% %   \tilde{\sigma} = \frac{2 \sigma_0}{N_R+1}\sqrt{\frac{N_R}{N_R+2}}.
% % \end{equation}
% % Similarly one can show that if
% % \begin{equation}
% %   R^*_\nu \approx \text{Unif}(0,2R_0),
% % \end{equation}
% % then
% % \begin{equation}
% %   E\left[\min_\nu \left(R^*_\nu\right)\right]  = \frac{2R_0}{N_R+1}.
% % \end{equation}
% % Then we can estimate
% % \begin{equation}
% %   \left(1-\max_\nu\left(\sigma_{i\nu}\right)\right)\min_\nu\left(R^*_\nu\right)N_R \gamma_0 \approx 2R_0\gamma_0\left(1-2\sigma_0 \frac{N_R}{N_R+1}\right)\frac{N_R}{N_R+1}
% % \end{equation}
% % We now find an upperbound for the RHS of Eq.\eqref{eq : dissipation constraint} :
% % \begin{equation}
% %   \sum_\nu \alpha_{\nu i}  \leq \sum_\nu \max_\nu\left(\alpha_{\nu i}\right) \approx 2\alpha_0 \frac{N_R}{N_R+1} N_R.
% % \end{equation}
% % This means that if we pick metaparameters verifying :
% % \begin{equation}
% %   R_0 \gamma_0 \left(1-2\sigma_0\frac{N_R}{N_R+1}\right) \gg \alpha_0 N_R.
% % \end{equation}
% % In the $N_R \gg 1$ limit :
% % \begin{equation}
% %   \frac{\gamma_0}{\alpha_0} \gg \frac{N_R}{R_0(1-2\sigma_0)}.
% % \end{equation}
% % This condition allows us to pick metaparameters that we know will satisfy the energy constraint most of the time. This is done solely to speed up computation time (energy constraint is checked anyway while building the system).
%
% \paragraph{Positivity of the parameters}
% Feasability means at least that every physical parameter defined here must be positive. In particular, this implies:
% \begin{equation}
%   d_i > 0 \implies \sum_\mu \sigma_{i\mu}\gamma_{i\mu}R^*_\mu > \sum_\mu \tau_{\mu i}
% \end{equation}
% If $\tau_{i\mu} = 0.$ this is trivially satisfied because $\sigma_{i\mu}$, $\gamma_{i\mu}$ and $R^*_\mu$ have all been drawn positive. However if $\tau_{\mu i} = \alpha_{\mu i}$ this is not always the case and we have to get parameters satisfying :
% \begin{equation}
%   \sum_\mu \sigma_{i \mu} \gamma_{i\mu}R^*_\mu > \sum_{\mu} \alpha_{\mu i}\text{ }\forall i.
% \end{equation}
% We can try to estimate the value of some metaparameters that would satisfy this.
% We have :
% \begin{equation}
%   \sum_\mu \sigma_{i\mu} \gamma_{i\mu} R^*_\mu \geq \mind{\sigma}\mind{\gamma}\mind{R^*}.
% \end{equation}
% Using this boundary and Eq.\eqref{eq : min and max energy constraint}, we know that $d_i>0$ if
% \begin{equation}
%   \maxd{\alpha}N_R \leq \mind{\sigma} \mind{\gamma}\mind{R^*},
% \end{equation}
% \ie
% \begin{equation}
%   \alpha_0 < \frac{\left(1-\epsilon\right)^3}{1+\epsilon} \frac{\sigma_0 \gamma_0 R_0}{N_R},
% \end{equation}
% or in the $\epsilon \ll 1$ limit :
% \begin{equation}
%   \alpha_0 < \left(1-4\epsilon\right)\frac{\sigma_0 \gamma_0 R_0}{N_R}.
% \end{equation}
% Similarly we must have a positive death rate for the resources, \ie:
% \begin{equation}
%   m_\nu = \frac{l_\nu-\sum_j \gamma_{j\nu} R^*_\nu S^*_j+\sum_j \alpha_{\nu j} S^*_j}{R^*_\nu} > 0. \label{eq : positive m_nu}
% \end{equation}
% This means we have to impose parameters that verify:
% \begin{equation}
%   l_\nu + \sum_j \alpha_{\nu j}S^*_j > \sum_j \gamma_{j \nu}R^*_\nu S^*_j.
% \end{equation}
% We can do a reasoning similar to before, \ie find a lower boundary for the LHS and an upper boundary for the RHS.
% We have
% \begin{equation}
%   l_\nu + \sum_{j} \alpha_{\nu j} S^*_j \geq \mind{l} + \mind{\alpha} \mind{S^*}
% \end{equation}
% and
% \begin{equation}
%   \sum_j \gamma_{j\nu}R^*_\nu S^*_j \leq N_S \maxd{\gamma}\maxd{R^*}\maxd{S^*}.
% \end{equation}
% Hence if we get parameters satisfying
% \begin{equation}
%   \mind{l}+\mind{\alpha}\mind{S^*} > N_S \maxd{\gamma}\maxd{R^*}\maxd{S^*},
% \end{equation}
% then Eq.\eqref{eq : positive m_nu} will be immediately satisfied. In terms of metaparameters this is equivalent to:
% \begin{equation}
%   \alpha_0 > \frac{N_S \gamma_0 R_0 S_0 (1+\epsilon)^3-l_0 (1-\epsilon)}{S_0 (1-\epsilon)^2}.\label{eq : alpha lowerbound 0}
% \end{equation}
% In the $\epsilon \ll 1$ limit this is equivalent to:
% \begin{equation}
%   \alpha_0 > \left(1+5\epsilon\right)N_S \gamma_0 R_0 - \left(1+\epsilon\right)\frac{l_0}{S_0}.
% \end{equation}
% (Interesting, if $l_0/S_0$ is large enough, \ie "there is enough food for everyone" then this condition is irrelevant).
% % Indeed,
% % \begin{equation}
% %   \sum_\mu \alpha_{\mu i} \leq \sum_\mu \max_\mu \left(\alpha_{\mu i}\right) \approx 2\alpha_0 \frac{N_R}{N_R+1} N_R \text{
% %     (see above).
% %   }
% % \end{equation}
% % Similarly,
% % \begin{equation}
% %   \sum_{\mu} \sigma_{i\mu}\gamma_{i\mu}R^*_\mu \geq \min_\mu\left(\sigma_{i\mu}R^*_\mu\right)\sum_\mu \gamma_{i\mu} \geq \min_\mu\left(\sigma_{i\mu}\right)\min_\mu\left(R^*_\mu\right) N_R \gamma_0.
% % \end{equation}
% % Using the estimations for $\min_\mu\left(\sigma_{i\mu}\right)$ and $\min_\mu\left(R^*_\mu\right)$ :
% %  \begin{equation}
% %    \min_\mu\left(\sigma_{i\mu}\right) \approx E\left[\min_\mu\left(\sigma_{i\mu}\right)\right] = \frac{2 \sigma_0}{N_R+1}\text{, }\min_\mu\left(R^*_\mu\right) \approx E\left[\min_\mu\left(R^*_\mu \right)\right] = \frac{2 R_0}{N_R+1}
% %  \end{equation}
% %  This means we have to take :
% %  \begin{equation}
% %    \frac{\gamma_0}{\alpha_0} \gg \frac{N_R(N_R+1)}{2\sigma_0 R_0}.
% %  \end{equation}
%
% \paragraph{Combining conditions}
% If we combine both upperbounds we get a restriction on the metaparameters:
% % \begin{equation}
% %   \frac{\gamma_0}{\alpha_0} \gg \max\left(\frac{N_R(N_R+1)}{2\sigma_0 R_0}, \frac{N_R}{R_0(1-2\sigma_0)}\right)
% % \end{equation}
% % To get an idea on the order of magnitude of the ratio $\gamma_0/\alpha_0$ (\ie our order parameter), if we work with $N_R = 25$, $\sigma_0 = 0.2$, $R_0 = 1$, we have
% % \begin{equation}
% %   \frac{N_R(N_R+1)}{2\sigma_0 R_0} = 1'625 \text{ and } \frac{N_R}{R_0(1-2\sigma_0)} \approx 42
% % \end{equation}
% % \ie we will need, if $\gamma_0 \approx 1$
% % \begin{equation}
% %   \alpha_0 \ll 6.1 \times 10^{-4}.
% % \end{equation}
% % That means only very small metabolite release (this is indeed what we observe numerically).
% \begin{subequations}\label{eq : alpha bounds}
% \begin{equation}
% \alpha_0 < \min\left(\frac{\left(1-\epsilon\right)^2}{1+\epsilon}\left(1-\left(1+\epsilon\right)\sigma_0\right)\frac{\gamma_0 R_0}{N_R}, \frac{\left(1-\epsilon\right)^3}{1+\epsilon} \frac{\sigma_0 \gamma_0 R_0}{N_R}\right). \label{eq : alpha upperbound}
% \end{equation}
% We of course also get a restriction on the lowerbound of $\alpha_0$ through Eq.\eqref{eq : alpha lowerbound 0}:
% \begin{equation}
%   \alpha_0 > \frac{N_S \gamma_0 R_0 S_0 (1+\epsilon)^3-l_0 (1-\epsilon)}{S_0 (1-\epsilon)^2}. \label{eq : alpha lowerbound}
% \end{equation}
% \end{subequations}
% To get an idea on the order of magnitude of $\alpha_0$ (which will be our order parameter if $\gamma_0=1$), we have for $N_R=25$, $\sigma_0 = 0.2$, $R_0 = 1$ and $\epsilon=0.1$ :
% \begin{equation}
%   \alpha_0 < 5.3 \times 10^{-3}.
% \end{equation}
% So what we see in Eq.\eqref{eq : alpha upperbound} is that $\alpha_0$ has an upper bound which is dictated either by energy conservation or system feasability. What relations do the metaparameters have to fulfill in these two different regimes?
%
% Suppose that the limiting factor is system feasability. That means:
% \begin{empheq}{align}
%    \frac{\left(1-\epsilon\right)^3}{1+\epsilon} \frac{\sigma_0 \gamma_0 R_0}{N_R} &\leq \frac{\left(1-\epsilon\right)^2}{1+\epsilon}\left(1-\left(1+\epsilon\right)\sigma_0\right)\frac{\gamma_0 R_0}{N_R} \nonumber \\
%   \iff  (1-\epsilon) \sigma_0 &\leq 1-(1+\epsilon)\sigma_0 \nonumber\\
%   \iff \sigma_0 &\leq \frac{1}{2}
% \end{empheq}
% This means if $\sigma_0 \leq \frac{1}{2}$, the limiting factor will be system feasability while if $\sigma_0 \geq \frac{1}{2}$, it will be energy conservation.


\end{document}
