\documentclass[12pt]{report}
\usepackage{consumer_resource_final}
\begin{document}
\subsection{Choice of metaparameters}
% Section \ref{sec : feasibility methods algorithmic procedure} explains the algorithmic procedure $\mathcal{A}(m, B) \in \mathcal{P}$ which allows us to build feasible parameters out of a set of metaparameters and a consumption-syntrophy network. However, in order to work properly, the combination of metaparameters used as an input of the algorithm must most of the time lead to the realisation of feasible systems. We hence need to find what region of the metaparameters space lead to a high feasibility : this is precisely the idea behind the notion of the feasibility region discussed below.

%But first, let's see how our study can made simpler.
Feasibility considerations discussed in Section \ref{sec : methods feasibility basic concepts} tell us that we have six metaparameters\footnote{Indeed, we saw that $d_i$ and $m_\mu$ are set by the other parameters, so we cannot freely choose $d_0$ and $m_0$ (see Section \ref{sec : methods feasibility basic concepts}).} which we can play with: $\gamma_0$, $\alpha_0$, $l_0$, $\sigma_0$, $S_0$ and $R_0$. Out of simplicity, we will fix three of these six and only $\gamma_0, S_0$ and $\alpha_0$ will be varied throughout this study. We hence need to fix the value of $R_0, l_0$ and $\sigma_0$ in a relevant way.

Following the analysis of \cite{barbier_cavity_2017}, we notice that our model Eqs.\eqref{eq: differential eq for resources and species} still possesses some freedom. Indeed we can choose the set of units we work in to fix the values of some metaparameters. There are two physical quantities at stake here: biomass and time, and we may choose, however we want it, a specific set of units describing both of them.
We measure biomass in units of the average resource abundance at equilibrium\footnote{That choice is not completely innocent. Indeed we see in other parts of the manuscript that the matrix $\alpha_{\nu i}-\gamma_{i \nu} R^*_\nu$ is a quantity crucial to the dynamics of the system. Setting $\av{R^*}=1$ allows us to simply study the impact of $\gamma$ against $\alpha$ instead of the more complicated $\gamma R^*$ versus $\alpha$.}:
\begin{equation}
 \av{R_\mu} = R_0 = 1.
\end{equation}
Similarly, we measure time such that the average external resource uptake rate is one, that is:
\begin{equation}
\av{l_\mu} = l_0 = 1.
\end{equation}
After this manipulation, only the value of $\sigma_0$ needs to be fixed. It is difficult to determine which value should be chosen because the efficiency of biomass production varies a lot according to the system in question \cite{delong_shifts_2010}. However we found analytically that because of physical considerations (Eq.\ref{eq : fully feasible volume}) feasibility depends not exactly on $\sigma_0$ but rather on $\min(1-\sigma_0, \sigma_0)$. This hints at the existence of two regimes, namely $\sigma_0 > 0.5$ and $\sigma_0 < 0.5$. Since in Nature efficiency is usually small \cite{delong_shifts_2010}, it is more realistic to be in the $[0, 0.5]$ regime. We will take the middle of that interval and choose $\sigma_0=0.25$.
%For the sake of simplicity, we keep the same $\sigma_0$ throughout our whole study.

Note also that when we generate parameters sets from metaparameters through the algorithmic procedure $\mathcal{A}$ (Section \ref{sec : feasibility methods algorithmic procedure}), we will take a small relative standard deviation, $\epsilon \approx 4 \%$, such that indeed the metaparameters approximation is justified.

As said above, $\alpha_0$, $\gamma_0$ and $S_0$ are the only metaparameters we will vary throughout this study. Because of that, we sometimes will elude the other metaparameters in the notation and will write instead of $m=(\gamma_0, S_0, \alpha_0, \sigma_0, R_0, l_0, d_0, m_0) \in \mathcal{M}$ simply $m=(\gamma_0, S_0, \alpha_0)$. It is explained in Section \ref{sec: impact of syntrophy on feasible region} which values of $\alpha_0$ we study. Concerning $\gamma_0$ and $S_0$, we consider most of the time points $(\gamma_0, S_0) \in [0.01, 1]\times[0.01, 1]$. We sometimes write the square $[0.01, 1]\times[0.01, 1]$ as $[0,1]^2$ or the unit square, where it is understood that points where either $\gamma_0$ or $S_0$ is smaller than $0.01$ are not taken into account.

\subsection{Set of matrices}\label{sec : set of matrices}
Throughout this Thesis we consider two sets of consumption matrices. The first one, called $G_{25}$, is made of sixty full-rank consumption matrices with the dimensions $N_R=N_S=25$. Except for one matrix with $\eta_G=0.15$ and $\kappa_G=0.18$, they are  distributed homogeneously among the $\eta_G>\kappa_G$ surface for $\eta_G \in [0.1, 0.6]$ and $\kappa_G \in [0.08, 0.43]$. The second set, called $G_{50}$, has dimensions $N_R=50$ and $N_S=25$ and comprises of forty-one full-rank consumption matrices, also homogeneously distributed among the surface $\eta_G>\kappa_G$ for the same ranges of $\eta_G$ and $\kappa_G$. All the matrices of this set are generated through the general MCMC algorithm\footnote{Technically that algorithm is designed to obtain an output matrix which is ``optimal'' for a given input matrix in the sense that it minimizes an energy that depends on both the output and input. By making the energy depend on the output only, we can generate our $G$-matrices with the same code.} explained in Section \ref{section: methods LRI MC solver}. The procedure to produce a matrix $G$ with connectance $\kappa_{\text{target}}$ and nestedness $\eta_{\text{target}}$ is simple. Indeed, by design, the algorithm allows us to choose the connectance of the output matrix and we can reach the desired nestedness by setting the energy as $E(G)=\abs{\eta(G)-\eta_{\text{target}}}$.

It is explained below that for each $G$-matrix we consider a small number of different $A$ syntrophy matrices. We denote a pair of consumption-syntrophy network $(G,A)$ sometimes as $B$, and for each scenario, the set of consumption-syntrophy networks $(G,A)$ is denoted\footnote{To avoid bloating the notation there is no mention of the $A$-scenario in the matrix set $S$ and it is made clear in the text what scenario of $A$ is chosen when we speak of $(G,A) \in S$.} as $S_{25}$ when $G \in G_{25}$ and as $S_{50}$ when $G \in G_{50}$.
In any case, all the matrices used in this study are available online at the address \url{https://gitlab.ethz.ch/palberto/consumersresources.git}.

\subsection{Syntrophy scenarios}\label{sec: syntrophy scenarios}
The general aim of this Thesis is to study how quantities that characterize feasibility and stability change with respect not only to the consumption matrix $\gamma$, but also to the syntrophy matrix $\alpha$. As explained above (Section \ref{sec : intro metaparameters and matrix properties}) the large complexity of this task makes us move to a statistical approach, where general matrices are separated in two parts, namely average non-zero link strength (what we call ``metaparameter'') and network structure. That led to the definition of the binary syntrophy network matrix $A$. We here will need to simplify our approach even more.

Indeed our goal is to study stability properties of many consumption-syntrophy networks $(G,A)$. We decide to focus on two sets of $G$-matrices varying two characteristics of $G$, namely connectance $\kappa_G$ and nestedness (which, for the case of $G$, we call ecological overlap) $\eta_G$. By symmetry we should also choose a set of $A$'s with different connectances and nestedness and study every possible pair $(G,A)$ together, making the total number of microbial communities to study equal to the multiplication of the number of consumption networks by the number of syntrophy networks. Working that way, although more scientifically thorough, would demand more time than what is allowed for this type of Thesis and is hence not possible. Because we still would like to study the effect of the shape of $A$, we decide, for each consumption matrix $G$, to consider four $A$ ``scenarios''\footnote{We talk here about scenarios because, apart from the FC case, $A$ will generally be different for each $G$. It is important to remember that even though we compare $A$-matrices ``in the same scenario'', those may have a very different shape.}:
\begin{itemize}
  \item ``Fully Connected'' (FC): $A$ is filled with ones only, $A_{\mu i}=1$. This corresponds to a so-called ``mean-field'' approximation. Every consumer releases every resource at the same intensity (up to some noise) .
  \item ``No Intraspecific Syntrophy'' (NIS): the structure of $A$ is such that consumers are not allowed to release what they consume, \ie there is no intraspecific syntrophy. Apart from that, they release everything else. $G$-matrices with a small connectance will have an $A$-matrix with a large connectance (not far away from the FC case) and vice-versa.
  \item ``Low Resource Interaction'' (LRI): $A$ is the outcome of the MCMC algorithm\footnote{Note that we will take a constant value $\alpha_0$ (given in Section \ref{section: methods LRI MC solver}) and $\gamma_0=0.2$. A more thorough analysis should build the optimal LRI matrix \important{corresponding to each $(\gamma_0, \alpha_0)$}. That would take too much time which is why we decided to keep $\gamma_0$ and $\alpha_0$ constant.}
 described in Methods \ref{section: methods LRI MC solver}. The purpose of this algorithm is to build an $A$ that minimizes the energy $E(G,A)$ (Eq.\ref{eq: dynamical stability methods LRI MC solver energy definition}), and hence get the $A$ which for a given $G$, brings the system as close to satisfaction of Eq.\eqref{eq: strong LRI regime} as possible. The connectance of the $A$-matrix is taken as the one of the $G$-matrix.
 \item ``Random Structure'' (RS) : $A$ is taken as a random matrix (with the right dimensions), which has the same connectance as the corresponding $G$ but where non-empty links have been randomly placed.
 \end{itemize}
Only these four cases are considered for the feasibility and dynamical stability sections. We consider more scenarios for the study of structural stability. These are described in Section \ref{sec: results structural stability critical dynamical syntrophies}.
\end{document}
