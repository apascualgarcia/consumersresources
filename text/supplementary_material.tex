\documentclass[12pt]{report}
\usepackage{consumer_resource_final}
\begin{document}
\subsection{Effective system}
Models which involve the dynamics of species only are in general better known than consumers-resources models [\textbf{insert reference}]. In particular, a huge body of literature exists on the study of Lotka-Volterra systems [\textbf{insert reference}]. We may profit from this knowledge by transforming the effect of the resources dynamics into an effective consumers-only system.

This can be done by assuming that the resources reach an equilibrium way faster than the consumers.
Mathematically, that is equivalent to
\begin{equation}
  \frac{dR_\mu}{dt} \approx 0, \\ \forall \mu.
\end{equation}
Using Eq.\eqref{eq: differential eq for resources}, we get an explicit value for the resources:
\begin{equation}
  R_\mu \approx \frac{l_\mu+\sum_j \alpha_{\mu j}S_j}{m_\mu + \sum_k \gamma_{k\mu}S_k}.
\end{equation}
This expression can be used in Eq.\eqref{eq: differential eq for species} to get an effective system which describes the dynamics of the $N_S$ consumers:
\begin{equation}
  \frac{dS_i}{dt} = \left(\sum_\nu \left(\frac{\sigma_{i\nu}\gamma_{i\nu}l_\nu}{m_\nu+\sum_k \gamma_{k\nu}S_k} - \alpha_{\nu i}\right) -d_i + \sum_{\nu j} \frac{\sigma_{i\nu}\gamma_{i\nu}\alpha_{\nu j}}{m_\nu+\sum_{k}\gamma_{k\nu}S_k}S_j \right) S_i.
\end{equation}
This can be rewritten in a more compact way:
\begin{equation}
  \frac{dS_i}{dt} = p_i(S) S_i + \sum_j M_{ij}(S)S_i S_j \label{eq: effective equations of evolution}
\end{equation}
with
\begin{equation}
    p_i(S) = -\left(d_i+\sum_{\nu}\alpha_{\nu i}\right) + \sum_\nu \frac{\sigma_{i\nu}\gamma_{i\nu}l_\nu}{m_\nu+\sum_k \gamma_{k\nu}S_k}\text{ and } M_{ij}(S)=\sum_{\nu}\frac{\sigma_{i\nu}\gamma_{i\nu}\alpha_{\nu j}}{m_\nu+\sum_{k}\gamma_{k\nu}S_k}.
\end{equation}
If we assume the species $S_k$ are not too far away from their equilibrium values\footnote{Note that this is very rarely true, especially in the context of the study of structural stability, where entire species sometimes die out.}, \ie
\begin{equation}
S_k \approx S^*_k \ \forall k,
\end{equation}
then using Eq.\eqref{eq: positive m_nu} we can simplify $p_i$. Indeed,
\begin{equation}
m_\nu + \sum_k \gamma_{k\nu} S_k \approx m_\nu + \sum_k \gamma_{k\nu}S^*_k = \frac{l_\nu + \sum_k \alpha_{\nu k}S^*_k}{R^*_\nu} \label{eq: equality fluxes resource}
\end{equation}
Hence, the explicit dynamical dependence on $S$ can be removed from $p_i$ and $M_{ij}$:
\begin{equation}
p_i(S) \approx p_i \equiv - \left(d_i + \sum_\nu \alpha_{\nu i}\right) + \sum_\nu \frac{\sigma_{i\nu}\gamma_{i\nu}l_\nu R^*_\nu}{l_\nu + \sum_k \alpha_{\nu k}S^*_k},
\end{equation} and
\begin{equation}
M_{ij}(S) \approx M_{ij} \equiv \sum_\nu \frac{\sigma_{i\nu} \gamma_{i\nu} R^*_\nu \alpha_{\nu j}}{l_\nu + \sum_k{\alpha_{\nu k} S^*_k}}.
\end{equation}
\subsubsection{Perturbation analysis}
We study a system that we put close to an equilibrium $S^*$, \ie
\begin{equation}
S=S^*+\Delta S, \\ \text{with } \Delta S \ll 1.
\end{equation}
Written this way, the effective equations of motion Eq.\eqref{eq: effective equations of evolution} are equivalent to:
\begin{equation}
\frac{d\Delta S_i}{dt} = p_i(S^*+\Delta S)\left(S^*_i + \Delta S_i\right)+\sum_j M_{ij}(S^*+\Delta S)\left(S^*_i +\Delta S_i\right)\left(S^*_j +\Delta S_j\right).
\end{equation}
Since the deviations from equilibrium $\Delta S_i \ll 1$, we can forget the terms in higher power than quadratic:
\begin{equation}
\frac{d\Delta S_i}{dt} = \tilde{p}_i \Delta S_i + \sum_j E_{ij} \Delta S_j + \bigO(\Delta S^2),\label{eq: effective equ evol at O(D2)}
\end{equation}
with
\begin{equation}
\tilde{p}_i \equiv p_i(S^*) + \sum_k M_{ik}(S^*)S_k^*, \label{eq: tilde p effective system}
\end{equation}
and
\begin{equation}
E_{ij} \equiv \left(\partiald{p_i}{S_j}\evaluatedat{S^*}+M_{ij}(S^*)+\sum_k \partiald{M_{ik}}{S_j}\evaluatedat{S^*}S^*_k\right)S^*_i.
\end{equation}
After some computations, we can get $\tilde{p}_i$ and $E_{ij}$ in terms of the initial parameters. Indeed,
\begin{equation}
p_i(S^*)= -\left(d_i + \sum_\nu \alpha_{\nu i}\right) + \sum_\nu \frac{\sigma_{i\nu}\gamma_{i\nu}l_\nu}{m_\nu + \sum_k \gamma_{k\nu}S^*_k}
\end{equation}
and
\begin{equation}
M_{ik}(S^*) = \sum_\nu \frac{\sigma_{i\nu}\gamma_{i\nu}\alpha_{\nu j}}{m_\nu + \sum_k \gamma_{k\nu}S^*_k}.
\end{equation}
Hence, using Eq.\eqref{eq: tilde p effective system}:
\begin{equation}
\tilde{p}_i = - \left(d_i + \sum_\nu \alpha_{\nu i}\right) + \sum_\nu \frac{\sigma_{i\nu}\gamma_{i\nu}}{m_\nu + \sum_k \gamma_{k\nu}S^*_k}\left(l_\nu+\sum_{j}\alpha_{\nu j} S^*_j\right).
\end{equation}
This can be simplified using Eq.\eqref{eq: equality fluxes resource} and Eq.\eqref{eq: equilibrium species}:
\begin{equation}
\tilde{p}_i=-d_i +\sum_\nu \sigma_{i\nu}\gamma_{i\nu}R^*_\nu = \sum_\nu \alpha_{\nu i}.
\end{equation}
With a similar computation, one finds
\begin{equation}
E_{ij}=\sum_\nu \frac{\sigma_{i\nu}\gamma_{i\nu}S^*_i}{m_\nu+\sum_k \gamma_{k\nu}S^*_k} \left(\alpha_{\nu j}-\gamma_{j\nu}R^*_\nu\right).
\end{equation}
Finally, Eq.\eqref{eq: effective equ evol at O(D2)} can be recast in
\begin{equation}
\frac{d\Delta S_i}{dt} = \sum_j (J_E)_{ij} \Delta S_j,
\end{equation}
where the effective $N_S\times N_S$ jacobian matrix $J_E$ is defined by:
\begin{equation}
(J_E)_{ij}=\sum_\nu \left[\frac{\sigma_{i\nu}\gamma_{i\nu}S^*_i}{m_\nu+\sum_k \gamma_{k\nu}S^*_k} \left(\alpha_{\nu j}-\gamma_{j\nu}R^*_\nu\right)+\alpha_{\nu i}\delta_{ij}\right].
\end{equation}
We see that we without surprise we find again the $\Beta, \Gamma $ and $\Delta$ matrices coming from the jacobian at equilibrium:
\begin{equation}
\left(J_E\right)_{ij}=\sum_\nu \left[\frac{\Beta_{i\nu}\Gamma_{\nu j}}{\Delta_\nu}+\alpha_{\nu i} \delta_{ij}\right]
\end{equation}

This matrix determines the stability of the equilibrium. Namely if the largest eigenvalue of $J_E$ is positive, the equilibrium is unstable. If it is negative, the equilibrium is stable. If it is zero, the equilibrium is marginal.

\end{document}
