\documentclass[12pt]{report}
\usepackage{consumer_resource_final}
\begin{document}
%\textbf{TO DO} put a short introduction here to explain what I am doing: after establishing the methods, we use them to get results. We first discuss feasibility, then two different types of stability which allow us to look at microbial communities from a different angle. Hard to do a continuous variation of $A$ for every $G$ considered so build it algorithmically -> following 4 schems
After establishing in the previous chapter the framework and methods we will use, we now work on getting results that are both biologically relevant and interpretable. We keep the same structure as Chapter \ref{chap : methods} and will focus first on the crucial but sometimes overlooked\footnote{The ``traditional'' approach over the course of years (see \eg \cite{may_will_1972}) to the study of ecological communities has been to study local dynamical stability through the means of random matrix theory. The idea is to study how the statistical properties of the jacobian matrix at equilibrium $J^*$ influence stability, with the assumption that $J^*$ is a random matrix. Only recently\textbf{Is this true?} have discussions arised \cite{allesina_stabilitycomplexity_2015, rohr_structural_2014} about in what limit, if any, that assumption is a good representation of Nature, especially since we discovered ecological communities are shaped by dynamics which may lead to non particular, non-random, structures \cite{bascompte_nested_2003, bastolla_architecture_2009, bonsall_life_2004, thebault_stability_2010}.} question of \important{feasibility}, which aims to determine what regimes of parameters and metaparameters give rise to microbial communities that could exist in Nature. We will then move on to study how microbial communities resist to perturbations, \ie how stable they are. Two types of perturbations will be considered. Determining how communities react to perturbations of the resources and microbial species abundances will lead us to consider their \important{local dynamical stability}. Finally, we will study how they react to environmental perturbations, \ie we will quantify their \important{structural stability}.

\textbf{TO DO : put the following in the future "global aims" section}
The aim of this thesis is to study how these quantities change with respect not only to the consumption matrix $\gamma$, but also to the syntrophy matrix $\alpha$. As explained above (Introduction \ref{sec : intro metaparameters and matrix properties}) the large complexity of this task made us move to a statistical approach, where general matrices are separated in two parts, namely average non-zero link strength (what we called metaparameter) and network structure. That led to the definition of the binary syntrophy network matrix $A$. We here will need to simplify our approach even more.

Indeed our goal is to study stability properties of many consumption-syntrophy networks $(G,A)$. We decided to focus on a set of $G$-matrices varying two characteristics of $G$, namely its connectance $\kappa_G$ and nestedness (which, for the case of $G$, we call ecological overlap) $\eta_G$. By symmetry we should also choose a set of $A$'s with different connectances and nestedness and study every possible pair $(G,A)$ together, making the total number of microbial communities to study equal to the multiplication of the number of consumption networks by the number of syntrophy networks. Working that way, although more scientifically thorough, would demand more time than what is allowed for this type of Thesis and is hence not possible. Because we still would like to study the effect of the shape of $A$, we decide, for each consumption matrix $G$, to consider four $A$ ``scenarios''\footnote{We talk here about scenarios because, apart from the FC case, $A$ will generally be different for each $G$. It is important to remember that even though we compare $A$-matrices ``in the same scenario'', those may have a very different shape.}:
\begin{itemize}
  \item ``Fully Connected'' (FC): $A$ is filled with ones only, $A_{\mu i}=1$. This corresponds to a so-called ``mean-field'' approximation. Every consumer releases every resource at the same intensity (up to some noise) .
  \item ``No Intraspecific Syntrophy'' (NIS): the structure of $A$ is such that consumers are not allowed to release what they consume, \ie there is no coprophagy. Apart from this, they release everything else. $G$-matrices with a small connectance will have an $A$-matrix with a large connectance (not far away from the FC case) and vice-versa.
  \item ``Low Resource Interaction'' (LRI): $A$ is the outcome of the MCMC algorithm\footnote{Note that we will take a constant value $\alpha_0$ (given in Methods \ref{section: methods LRI MC solver}) and $\gamma_0=0.2$. A more thorough analysis should build the optimal LRI matrix \important{corresponding to each $(\gamma_0, \alpha_0)$}. That would take too much time which is why we decided to keep $\gamma_0$ and $\alpha_0$ constant.}
 described in Methods \ref{section: methods LRI MC solver}. The purpose of this algorithm is to build an $A$ that minimizes the energy $E(G,A)$ (Eq.\ref{eq: dynamical stability methods LRI MC solver energy definition}), and hence get the $A$ which for a given $G$, brings the system as close to satisfaction of Eq.\eqref{eq: strong LRI regime} as possible. The connectance of the $A$-matrix is taken as the one of the $G$-matrix.
 \item ``Random Structure'' (RS) : $A$ is taken as a random matrix (with the right dimensions), which has the same connectance as the corresponding $G$ but where non-empty links have been randomly placed.
 \end{itemize}
These four scenarios represent very different regimes, which, we hope, will have an impact on the feasibility or stability of the microbial communities considered.
\end{document}
