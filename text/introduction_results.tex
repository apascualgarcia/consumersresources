\documentclass[12pt]{report}
\usepackage{consumer_resource_final}
\begin{document}
%\textbf{TO DO} put a short introduction here to explain what I am doing: after establishing the methods, we use them to get results. We first discuss feasibility, then two different types of stability which allow us to look at microbial communities from a different angle. Hard to do a continuous variation of $A$ for every $G$ considered so build it algorithmically -> following 4 schems
After establishing in the previous chapter the framework and methods we will use, we now work on getting results that are both biologically relevant and interpretable. We keep the same structure as Chapter \ref{chap : methods} and will focus first on the crucial but sometimes overlooked\footnote{The ``traditional'' approach over the course of years (see \eg \cite{may_will_1972}) to the study of ecological communities has been to study local dynamical stability through the means of random matrix theory. The idea is to study how the statistical properties of the jacobian matrix at equilibrium $J^*$ influence stability, with the assumption that $J^*$ is a random matrix. Only recently\textbf{Is this true?} have discussions arised \cite{allesina_stabilitycomplexity_2015, rohr_structural_2014} about in what limit, if any, that assumption is a good representation of Nature, especially since we discovered ecological communities are shaped by dynamics which may lead to non particular, non-random, structures \cite{bascompte_nested_2003, bastolla_architecture_2009, bonsall_life_2004, thebault_stability_2010}.} question of \important{feasibility}, which aims to determine what regimes of parameters and metaparameters give rise to microbial communities that could exist in Nature. We will then move on to study how microbial communities resist to perturbations, \ie how stable they are. Two types of perturbations will be considered. Determining how communities react to perturbations of the resources and microbial species abundances will lead us to consider their \important{local dynamical stability}. Finally, we will study how they react to environmental perturbations, \ie we will quantify their \important{structural stability}.
\end{document}
