\documentclass[12pt, titlepage]{report}
\usepackage{consumer_resource_final}
\graphicspath{{./figures/}}

\begin{document}

We deal here with the last question of this thesis, which is the one of \important{structural stability}. We follow the path opened by \citeauthor{bastolla_architecture_2009} in the Supplementary Material of \cite{bastolla_architecture_2009} and later explored by  \citeauthor{rohr_structural_2014} in \cite{rohr_structural_2014}. The latter defined the question of structural stability as \textit{``asking how large is the range of parameter values that are compatible with the stable coexistence of all species''} \cite{rohr_structural_2014}. Their analysis was done in the context of the study of plants-pollinators networks.
For microbial communities, apart from incipient considerations found in  \cite{tikhonov_collective_2017,marsland_available_2019}, only \citeauthor{butler_stability_2018} fully considered that question \cite{butler_stability_2018} and even found a sufficient condition for structural stability of their model under some special assumptions\footnote{The model they consider in \cite{butler_stability_2018} differs slightly from ours (\eg resources do not die out $m_\mu=0$) and the sufficient condition they found assumes fully specialist consumers \ie $N_R=N_S$ and $\gamma=\gamma_0 \identity{N_S}$.}.

In the following pages, we investigate how the \important{critical structural perturbation} $\Delta_S^*(m, G, A)$, which measures for a given set of metaparameters $m$ and a consumption-syntrophy network $(G,A)$ leading to dynamically stable systems how much the external resources feeding rate $l_0$ may be altered without starting to observe microbial extinctions, changes as a function of the syntrophy $\alpha_0$ and the structure of both $G$ and $A$. We aim to find what topologies and syntrophic interactions lead to systems that can sustain the largest perturbations.
\subsection{Domain of analysis}
Appendix \ref{app: results structural stability procedure to estimate critical structural perturbation} explains the numerical algorithm we designed to determine $\Delta_S^*(m, G, A)$. Although this algorithm always comes to an end, it does not do it fast: typically it takes on the order of $\sim 1$ hour to provide its result. This means we cannot analyze in detail the whole fully dynamically stable domain $\mathcal{D}_{L,1}^{G,A}(\alpha_0)$ studied earlier and need to focus on some special, most interesting points.

As explained before, in order to assess the structural stability of a given parameter set $p$, we require that $p$ is locally dynamically stable \textbf{say why?}. Hence we need to work with metaparameters $m$ that are in a highly dynamically stable region. Furthermore, in order for comparisons to make sense, $m=(\gamma_0, S_0, \alpha_0)$ should be in a highly dynamically stable region \important{for every consumption-network}, which is why we choose to study points $m=(\gamma_0, S_0, \alpha_0) \in \mathcal{D}_{L,1}^{G,A}(\alpha_0)$. This gets however a bit tricky: as $\alpha_0$ increases, the shape of $\mathcal{D}_{L,1}^{G,A}(\alpha_0)$ changes! Since we would like to study how modifying \important{the syntrophy only} while letting the other parameters fixed impacts the system, we need to choose $(\gamma_0, S_0)$ such that
$(\gamma_0, S_0, \alpha_0) \in \mathcal{D}_{L,1}^{G,A}(\alpha_0)$ for all values of $\alpha_0$ we want to investigate. The idea is of course to take $\alpha_0$ as large as possible. Figure \ref{fig: dynamical stability results common fully dynamically stable volume} shows that for the case $N_R=N_S=25$, the largest syntrophy for which $\mathcal{D}_{L,1}^{S_M}(\alpha_0)$ is not empty is smaller than \num{1.3e-3}, which is only a tenth of the largest common syntrophy $\sim 0.01$ (Eq.\ref{eq : results feasibility largest alpha0})! Since the case $N_R=N_S=25$ \important{a priori} will not provide significant results, we turn to matrices with $N_R > N_S$, which according to literature leads to more stable systems \cite{biroli_marginally_2018}. Indeed Figure \ref{fig: dynamical stability results common lds volume NR=50 NS=25} shows that for $N_R=50$ and $N_S=25$, $\mathcal{D}_{L,1}^{S_M}(\alpha_0)$ is not empty until at least $\alpha_0 \approx \num{3.9e-3}$. This indicates that such systems will be more dynamically stable, which is why we choose to work with the set of matrices $S_{50}$ instead of $S_{25}$ \textbf{see if name is okay}. Looking at Figure \ref{fig: dynamical stability results common lds volume NR=50 NS=25}, we see that points with $\gamma_0 \gtrapprox 0.7$ can sustain the largest syntrophy while remaining fully dynamically stable for all matrices. In consequence we choose $(\gamma_0, S_0)=(0.75, 0.05)$ and keep these fixed until the end of this section.

Now that we chose $\gamma_0, S_0, N_R$ and $N_S$ such that we can work with a fairly high syntrophy, we still need to decide for which values of $\alpha_0$ we compute $\Delta_S^*(m, G, A)$. For a fixed $\gamma_0, S_0, G$ and $A$, we define the \define{critical dynamical syntrophy} $\alpha_C^D(\gamma_0, S_0, G, A)$ as:
\begin{equation}
 \alpha_C^D(\gamma_0, S_0, G, A)\defined \max_{\alpha_0}\left\{\alpha_0 : \mathcal{D}_{L,1}\left((\gamma_0, S_0, \alpha_0), G, A\right)=1\right\}.
\end{equation}
In words, $\alpha_C^D(\gamma_0, S_0)$ is the largest syntrophy for which we are sure that a system built with the procedure $\mathcal{A}$ is locally dynamically stable. As is explained in the next section, $\alpha_C^D(\gamma_0, S_0)$ depends heavily on both the structure of $G$ and $A$ and it definitely cannot be approximated as the same for all matrices considered. Since we want to get noticeable effects on $\Delta_S^*(m, G, A)$, we will compute it for each network at its individual critical syntrophy. To add  another point of comparison, we will also compute it for each at the lowest critical syntrophy found, which is the largest syntrophy which leads to fully dynamically stable systems for all networks. Finally, we will compare these two $\Delta_S^*$ with the one obtained when there is no syntrophy, \ie $\alpha_0=0$, which will act as a null model.

\subsection{Critical dynamical syntrophies}\label{sec: results structural stability critical dynamical syntrophies}
Figure \ref{fig: critical dynamical syntrophy fully connected syntrophy} shows how $\alpha_C^D(\gamma_0, S_0, G, A)$ evolves for the case of $A$ fully connected as a function of connectance $\kappa_G$ and ecological overlap $\eta_G$ of the consumption matrix. We observe a clear trend, in accordance with prior results: at fixed ecological overlap, networks with a larger connectance can attain more syntrophy while remaining dynamically stable and at fixed connectance, networks with a larger ecological overlap become unstable faster as syntrophy increases. Apart from the four $A$ structures considered since the beginning of this Thesis (FC, NIS, LRI, RS), we include three additional $A$-topologies:
\begin{itemize}
\item No Intraspecific Syntrophy Consumption matrix Connectance (NISCC); $A$ is random, except that no intraspecific syntrophy is allowed. Its connectance is taken as the connectance of $G$. This is more or less a  ``lower connectance version'' of the NIS scenario.

\item LRI matrix with NIS Connectance (LNISC); $A$ is the outcome of the LRI MCMC procedure described in Methods \ref{section: methods LRI MC solver}, except that in contrast to the LRI scenario where $\kappa_A=\kappa_G$, the connectance of $A$ is taken as the one of $A$ in the NIS scenario.

\item Random Structure with NIS Connectance (RNISC); $A$ has a completely random structure. Its connectance is chosen as the connectance of $A$ in the NIS regime.
\end{itemize}
Figure \ref{fig : ss results critical dynamical syntrophy different A same conn} shows how $\alpha_C^D$ changes as a function of the topology of $A$. The NISCC and RS outperforms the other scenarios for every consumption matrix considered. Both the FC and NIS cases have $\kappa_A$ larger than the RS scenario, which hints that systems where many syntrophic interactions take place (\ie $A$ has a large connectance) can sustain an overall smaller maximal syntrophic strength. The main difference between the LRI and the NISCC/RS regimes, since their respective syntrophy matrix have the same connectance, is their \important{syntrophic overlap}, \ie the nestedness of $A$, as is shown by Figure \ref{fig: dynamical stability results nestedness LRI outcome}. Although NISCC and RS both have the same connectance and an approximately similar syntrophic overlap, the main difference is that NISCC does not allow intraspecific syntrophy. In the end it seems like dynamical stability is favoured by the following three factors: low connectance of $A$, low syntrophic overlap of $A$ and prohibition of intraspecific syntrophy. Microbial communities where consumers do not release too many resources -- and if they do, in separate niches -- can achieve a larger average syntrophy than others while remaining dynamically stable.
% For a consumption-syntrophy network $(G,A)$ and metaparameters $m$, one can define the \important{critical structural perturbation} $\Delta_S^*(m, G, A)$ (see Methods \ref{sec : structural stability methods numerical estimate critical perturbation}). $\Delta_S^*(m, G, A)$ $\Delta_S^*(m, G, A)$ can be computed easily by solving the equation $P_E(\Delta_S^*(m, G, A), m, G, A)=0.5$ : at the critical structural perturbation, the chance of observing at least one microbial extinction is $0.5$ (see Appendix \ref{app: results structural stability procedure to estimate critical structural perturbation}).

\subsection{Critical structural perturbation}
Now that we calculated the critical dynamical syntrophies of each consumption-syntrophy network, we compare their critical structural perturbation $\Delta_S^*(m, G, A)$. As a ``null model'', we first compute $\Delta_S^*$ when there is no syntrophy at play. Figure \ref{fig: ss results critical delta no syntrophy} shows that structural stability confirms the trend hithertho observed: for a given ecological overlap $\eta_G$, $\Delta_S^*$ increases as the connectance $\kappa_G$ increases and for a given $\kappa_G$, $\Delta_S^*$ decreases as $\eta_G$ increases. In short, microbial communities where microbes consume a lot of different resources but do not share them resist best to environmental perturbations. These results coincide well with intuition.
%Indeed imagine a community where each microbe eats only very few resources (low $\kappa_G$).
When a system is structurally perturbed, the external resources feeding rates get shuffled\footnote{By ``shuffling'', we mean that the $l_\mu$ change but their average value does not. Indeed, recall that the $\nu_\mu$ in Eq.\eqref{eq : ss methods l perturbation}, have a zero average value which implies that after the perturbation $l_0$ remains the same, and so should the other metaparameters.}, which in turn shuffles the resources available for microbial consumption: some of them start becoming more abundant, some of them get scarcer. If a given microbial species only eats a small number of resources, by luck it is possible that most of its resources got rarer after the environmental perturbation such that the biomass it can eat is not large enough for its survival anymore and it is driven to extinction. On the other hand, if said microbial species eats from many resources, it is unlikely that all the resources it consumes got scarcer after the system perturbation. The lack of biomass from the scarcer resources should indeed be compensated by the additional biomass coming from the more abundant resources, which makes the species less prone to extinction. Having a larger connectance means that on average species consume more resources, which makes the system more stable and at a given connectance, having a larger ecological overlap means that the consumption will have a more triangular shape, which implies there will be some species that eat very few resources which makes the system unstable.

We now focus on what happens when the system is syntrophic, \ie $\alpha_0 > 0$. Figure \ref{fig: ss results critical delta deviation from no syntrophy} shows that surprisingly for the FC, LRI and RS cases, we observe no significant deviation away from the ``no syntrophy'' case. Whether $\Delta_S^*(m, G, A)$ is computed at each individual $\alpha_C^D$ or at $\min\{\alpha_C^D\}$, it seems like syntrophy does not influence much the structural stability of the system, at least not in a clearly decidable way\footnote{Some effect is sometimes indeed observed but the errors are so large compared to the magnitude of the effect that this very well could be noise.} The only scenario where a clear effect is observable is for $\Delta_S^*$ computed at the critical dynamical syntrophy in the NIS regime. There lies an interesting fact: even though an $A$ with a random structure gives the system a larger critical dynamical syntrophy, so in a sense a larger dynamical stability \textbf{is this really true}, it does not give it a larger structural stability. We observe a trade-off in the structure of the syntrophy matrix: if you want a larger syntrophy while remaining dynamically stable, you have to give up the fact that it will reinforce your structural stability. On the other hand if you want a configuration such that the syntrophy increases structural stability, the syntrophic strength is not the largest your consumption matrix could theoretically achieve.
Benefit gets higher as connectance increases


\FloatBarrier
\begin{figure}
\captionsetup[subfigure]{captionskip=-189pt, margin=50pt}
\hspace{-0.1\linewidth}
\subfloat[]{\includegraphics[width=0.6\linewidth]{{critical_dynamical_syntrophy_fixed_nest_FC}.pdf}}
\subfloat[]{\includegraphics[width=0.6\linewidth]{{critical_dynamical_syntrophy_fixed_conn_FC}.pdf}}
\caption{Critical dynamical syntrophy $\alpha_C^D$ for all the consumption matrices $G$ in the set $S_{50}$ (fully connected syntrophy matrix) (a) as a function of the connectance of $G$ for fixed ecological overlap and (b) as a function of the ecological overlap of $G$ for fixed connectance. For a fixed ecological overlap, systems with a  larger connectance can attain larger syntrophies. For a fixed connectance, a small ecological overlap is needed to get a large critical dynamical syntrophy.}\label{fig: critical dynamical syntrophy fully connected syntrophy}
\end{figure}
\begin{figure}
\captionsetup[subfigure]{captionskip=-203pt, margin=45pt}
\hspace{-0.1\linewidth}
\subfloat[]{\includegraphics[width=0.6\linewidth]{{critical_dynamical_syntrophy_Conn0.18}.pdf}}
\subfloat[]{\includegraphics[width=0.6\linewidth]{{critical_dynamical_syntrophy_Conn0.43}.pdf}}
\caption{Critical dynamical syntrophy $\alpha_C^D(\gamma_0=0.75, S_0=0.05, G, A)$ as a function of the ecological overlap of the consumption matrix $G$ for fixed connectance (a) $\kappa_G=0.18$ and (b) $\kappa_G=0.43$. The different lines symbolize the different structures of the syntrophy matrix $A$: FC, NIS, LRI, RS but also RNISC, LNISC, and NISCC (which are explained in the main text). A higher critical dynamical syntrophy is achieved when $A$ has a random structure. \textbf{because lower connectance??}}\label{fig : ss results critical dynamical syntrophy different A same conn}
\end{figure}
\begin{figure}
\hspace{-0.1\linewidth}
\captionsetup[subfigure]{captionskip=-190pt, margin=46pt}
\subfloat[]{\includegraphics[width=0.6\linewidth]{{critical_delta_str_stab_fixed_nest_no_syntrophy}.pdf}}
\subfloat[]{\includegraphics[width=0.6\linewidth]{{critical_delta_str_stab_fixed_conn_no_syntrophy}.pdf}}
\caption{Critical structural perturbation without syntrophy $\Delta_S^*(m,G,A=0)$ (a) as a function of ecological overlap with fixed connectance and (b) as a function of connectance for a fixed ecological overlap. We look at matrices with $N_R=50$ and $N_S=25$ at one of the points in the metaparameters space that are the most dynamically stable for all the matrices (see Fig.\ref{fig: dynamical stability results common lds volume NR=50 NS=25}), namely $(\gamma_0, S_0)=(0.75, 0.05)$. A clear trend may be observed, which is coherent with what was seen in Figure \ref{fig: dynamical stability results critical dynamical syntrophy}: for a given connectance, communities with a large ecological overlap are structurally less stable. Similarly, for a given ecological overlap, microbial communities with a consumption matrix with a larger connectance are more structurally stable.}\label{fig: ss results critical delta no syntrophy}
\end{figure}
\begin{figure}
\captionsetup[subfigure]{captionskip=-190pt, margin=63pt}
\hspace{-0.1\linewidth}
\subfloat[]{\includegraphics[width=0.6\linewidth]{{critical_delta_deviation_from_no_syntrophy_str_stab_fixed_conn_own_max_syntrophy_NISCC}.pdf}}
\subfloat[]{\includegraphics[width=0.6\linewidth]{{critical_delta_deviation_from_no_syntrophy_str_stab_fixed_nest_common_max_syntrophy_fully_connected}.pdf}}
\caption{Typical deviation away from the ``no syntrophy'' scenario. For all structures of the syntrophy matrix considered, apart from NIS (see Fig.\ref{fig: ss results own synt NIS critical delta deviation from no syntrophy}), adding a large syntrophy, be it the ``common'' or ``individual'' maximum, does not significantly increase the structural stability of the microbial community.}
\label{fig: ss results critical delta deviation from no syntrophy}
\end{figure}

\begin{figure}
\captionsetup[subfigure]{captionskip=-190pt, margin=60pt}
\hspace{-0.1\linewidth}
\subfloat[]{\includegraphics[width=0.6\linewidth]{{critical_delta_deviation_from_no_syntrophy_str_stab_fixed_nest_own_max_syntrophy_no_release_when_eat}.pdf}}
\subfloat[]{\includegraphics[width=0.6\linewidth]{{critical_delta_deviation_from_no_syntrophy_str_stab_fixed_conn_own_max_syntrophy_no_release_when_eat}.pdf}}
\caption{Deviation from the ``no syntrophy'' case for the syntrophy matrix $A$ with a NIS structure. $\alpha_0$ is taken at the critical dynamical syntrophy $\alpha_C^D$ for each system. At the largest possible syntrophy,  }
\label{fig: ss results own synt NIS critical delta deviation from no syntrophy}
\end{figure}

\begin{figure}
\captionsetup[subfigure]{captionskip=-202pt, margin=62pt}
\hspace{-0.1\linewidth}
\subfloat[]{\includegraphics[width=0.6\linewidth]{{critical_delta_difference_from_null_case_Conn0.18}.pdf}}
\subfloat[]{\includegraphics[width=0.6\linewidth]{{critical_delta_difference_from_null_case_Conn0.43}.pdf}}
\caption{Difference between the critical structural perturbation with and without syntrophy for different scenarios as a function of the ecological overlap for a given connectance (a) $\kappa_G=0.18$ and (b) $\kappa_G=0.43$. The different markers correspond to different $\alpha_0$ taken : triangle is ``common maximum syntrophy'', circle is ``own individual syntrophy'' and square is no syntrophy. The different colours correspond to different structures of $A$ : blue is FC, green NIS, red LRI, black RS, orange RNISC, pink LNISC and grey NISCC. At high connectance, the ``own individual syntrophy'' NIS scenario is $\sim 10 \%$ more structurally stable than the ``no syntrophy'' case.}\label{fig: ss results critical delta deviation from no syntrophy fixed conn varying nest}
\end{figure}




\end{document}
