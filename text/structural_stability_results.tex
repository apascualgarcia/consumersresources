\documentclass[12pt, titlepage]{report}
\usepackage{consumer_resource_final}
\graphicspath{{./figures/}}

\begin{document}
\subsection{Estimating the critical structural perturbation}\label{sec: results structural stability procedure to estimate critical structural perturbation}
\begin{figure}
\includegraphics{{typical_probability_structural_stability_curve}.pdf}
\caption{Typical probability of finding one extinction when structurally perturbing the system with a magnitude $\Delta_S$. The critical structural perturbation is easily estimated with a sigmoidal fit.}
\end{figure}


\begin{figure}
\hspace{-0.1\linewidth}
\captionsetup[subfigure]{captionskip=-190pt, margin=54pt}
\subfloat[]{\includegraphics[width=0.6\linewidth]{{critical_structural_perturbation_NR25_NS25_NR25.0NS25.0_nest_fixed_conn_all_points}.pdf}}
\subfloat[]{\includegraphics[width=0.6\linewidth]{{critical_structural_perturbation_NR25_NS25_NR25.0NS25.0_conn_fixed_nest_all_points}.pdf}}
\caption{Critical structural perturbation $\Delta_S^*(m,G,A)$ (a) as a function of ecological overlap with fixed connectance and (b) as a function of connectance for a fixed ecological overlap. The critical perturbations are not very high, of the order of one percent. A clear trend may be observed, which is coherent with what was seen in Figure \ref{fig: dynamical stability results critical dynamical syntrophy}: for a given connectance, communities with a large ecological overlap are structurally less stable. Finding a trend for how $\Delta_S^*$ varies as for a given ecological overlap as the connectance varies is harder because of the large errors.}
\end{figure}

\end{document}
