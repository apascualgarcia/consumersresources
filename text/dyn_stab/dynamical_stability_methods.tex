\documentclass[12pt, titlepage]{report}
\usepackage{consumer_resource_final}
\graphicspath{{./figures/}}

\begin{document}
\subsection{The master equation for local dynamical stability}\label{sec : establish master equation for dynamical stability}
In order to get $\real{\lambda_1}$, which is the quantity that governs local dynamical stability (see Section \ref{sec : how to determine local dynamical stability}), we have to get the full spectrum of $J^*$, denoted $\sigma(J^*)$, since a straight forward application of easier standard techniques like the Perron-Frobenius theorem \cite{perron_zur_nodate} does not work. The eigenvalues of $J^*$ are obtained through the eigenvalue problem:
\begin{equation}
\det\left(J^* - \lambda \right) = 0.
\end{equation}
More explicitly, using Eq.\eqref{eq: jacobian at equilibrium}, we state the \important{master equation for local dynamical stability}:
\begin{equation}
\boxed{
\det
\begin{pmatrix}
 -D - \lambda  & \Gamma \\
 \Beta & 0-\lambda
\end{pmatrix} = 0
}
\end{equation}
% Before proceeding any further, we eliminate systems where local dynamical stability cannot be decided with the first order perturbation analysis that we are conducting, \ie marginally stable equilibria.
That equation is not trivially solved, which is why we seek regimes where it could be made simpler.
% \subsubsection{Marginally stable equilbria}
% We want to avoid the case $\real{\lambda_1}=0$, because it does not let us assess the system's stability without a further complicated mathematical analysis. To make things easier, we will in analytical computations\footnote{For numerical computations we will get rid of marginally stable systems individually, meaning we may have some systems where $0 \in \sigma(J^*)$.} get rid of all systems where 0 is part of the spectrum. Even though this is a harsh condition, we know that for such systems, local dynamical stability can be decided by the computations we conduct. $\lambda=0$ is part of the spectrum if and only if it solves the master equation \eqref{eq: master equation eigenvalues jacobian}:
% \begin{equation}
% \det
% \begin{pmatrix}
%   -\Delta   & \Gamma \\
%   \Beta & 0
% \end{pmatrix} = 0 \label{eq: determinant marginally stable equilibria}
% \end{equation}
% Using the fact that $\Delta$ is invertible, we can make use of the equality\footnote{This uses a formula which is trivially analogous to one found in \cite{powell_calculating_2011}.}:
% \begin{equation}
% \det\begin{pmatrix}
%   -\Delta   & \Gamma \\
%   \Beta & 0
% \end{pmatrix} = \det(-\Delta)\det(\Beta\Delta^{-1}\Gamma).
% \end{equation}
% Eq.\eqref{eq: determinant marginally stable equilibria} then becomes:
% \begin{equation}
% \det(\Beta\Delta^{-1}\Gamma)=0
% \end{equation}
% which means that $\Beta \Delta^{-1}\Gamma$ is not full rank. We can hence state the following


\subsection{Bounds on the eigenvalues: Gerschgorin circle theorem}\label{sec : gerschgorin circle theorem}
% Before studying Equation \eqref{eq: master equation non marginal}, we would like to know more about the spectrum of $J^*$. The most critical question is knowing \important{where} we expect the eigenvalues of $J^*$ to lie on the complex plane.
% \subsubsection{Gerschgorin circle theorem}
Gerschgorin circle theorem \cite{gerschgorin_uber_1931}
%allows us to get a better idea of the location of the eigenvalues in the complex plane. It
states that every eigenvalue of a $N\times N$ square matrix $A$
is located in one of the $N$ discs $\tilde{D}_i$ defined by:
\begin{equation}
\tilde{D}_i\defined \left\{z \in \mathbb{C}: \abs{z-A_{ii}} \leq \sum_{j\neq i} \abs{A_{ij}} \right\}. \label{eq: definition discs Gerschgorin}
\end{equation}
In a more mathematical language:
\begin{equation}
\sigma(A) \subset \union_{i=1}^N \tilde{D}_i, \label{eq: circle theorem}
\end{equation}
where $\sigma(A)$ is the spectrum of $A$. Intuitively, the circle theorem tells us that the eigenvalues of a matrix deviate from the diagonal elements by a value bounded by the sum of the off-diagonal elements.
It is easy to see that if all the discs $\tilde{D}_i$ are located to the left of the imaginary axis (\ie the discs contain only numbers with a negative real part), then the eigenvalues of $A$ are all negative.
% \subsection{Low intra resources interaction (LRI) regime }
% Now that we have a bound on how big the eigenvalues can be, we need strategies to find regimes where we \important{know} $\real{\lambda_1} < 0 $, \ie local dynamical stability is guaranteed. We inspire ourselves from the general idea of the mathematical proofs of \cite{butler_stability_2018}.
\subsection{Reductio ad absurdum} \label{subsubsec: reductio ad absurdum}
The strategy we use to solve Eq.\eqref{eq: eigenvalue problem S} is inspired by \cite{butler_stability_2018}. The strategy is to assume we are in an unstable regime, \ie there exists at least one $\lambda \in \sigma(J^*)$ with $\real{\lambda}\geq 0$ that satisfies Eq.\eqref{eq: master equation non marginal} and such that $\real{\lambda} > 0$. By Eq.\eqref{eq: eigenvalue problem S}, $\lambda$ is also an eigenvalue of $S(\lambda)$. If we find conditions under which the real part of the spectrum of $S(\lambda)$ is entirely negative, we will know that $\real{\lambda} \leq 0$. As this is a contradiction to the hypothesis that the regime is unstable, we must conclude that the regime is stable\footnote{Indeed, Eq.\eqref{eq: master equation non marginal} assumes already that either $\real{\lambda_1} >0$ or $\real{\lambda_1} < 0$.}.

Hence, the general idea is to find regimes where we know that the spectrum of $S$%, written as $\sigma(S)$,
will be entirely negative for a positive $\lambda$. Thanks to the help of the two following theorems, we found such a regime, called \define{low intra-resource interaction} or LRI.


\subsection{Monte Carlo algorithm for the optimal matrix} \label{section: methods LRI MC solver}
We explain how to write a Metropolis-Hastings Markov Chain Monte Carlo (MCMC) method which for an input matrix $M_I$ gives as an output the optimal matrix $M_O$ which minimizes the energy $E(M_I, M_O)$. The procedure follows a traditional Metropolis algorithm:
\begin{enumerate}
\item Create a random binary $M_O$. Its connectance is chosen as the one of the consumption matrix $G$.
\item Do the following for a given number of steps:
\begin{itemize}
\item Choose a random row or, every other iteration, a column.
\item In that row/column, try to swap a zero and a one while preserving the ``releasers'': if a species releases some resource, it has to keep releasing something (the resource can change though). The ``releasees'' are preserved as well: if a resource is being released by some species, it has to keep being released (but it does not have to be by the same species).
% REMARK : why do we impose such conditions?
\item The swap is accepted, \ie $M_O$ is modified, if the energy difference $\Delta E$ is negative or if a random number drawn uniformly between zero and one is smaller than $e^{-\Delta E/T}$ where $T$ is the current temperature. More on $\Delta E$ and $T$ below.
\end{itemize}
\item Return $M_O$.
\end{enumerate}
A couple comments on this algorithm can be made:
\begin{itemize}
\item The algorithm preserves the connectance of $M_O$ but not its nestedness. The question of what connectance to choose is open, but we choose $\kappa(M_O)=\kappa(M_I)$ as a first approach, \ie input and output matrices have the same connectance.
\item The temperature $T$ changes dynamically during the simulation. It is obtained in a way close to the spirit of simulated annealing techniques \cite{gendreau_simulated_2019}: the temperature $T$ is multiplied by a factor $\lambda=0.99$ at a fixed frequency (for instance every 1000 steps). We add the requirement that if new moves are rejected during too many consecutive steps, we multiply the temperature by $1/\lambda$.
\item The energy difference $\Delta E$ between the new proposed $M_O'$ and the old $M_O$ is computed by assigning an energy $E$ to both $M_O'$ and $M_O$ and subtracting them:
\begin{equation}
\Delta E \defined E(M_I, M_O')-E(M_I, M_O).
\end{equation}
%The choice of the energy function $E$ is crucial. By design, this MCMC algorithm will find the specific $M_O$ which minimizes $E(M_I, M_O)$.
\end{itemize}
That algorithm is used not only to generate the $G$ consumption matrices (Section \ref{sec : set of matrices}), in that case $E$ only depends on the output $G$ so the input matrix does not matter, but also to find the optimal ``LRI'' matrix. In that situation, $E$ depends both on the input consumption matrix $G$ and the output syntrophy matrix $A$ and is given by Eq.\eqref{eq: dynamical stability methods LRI MC solver energy definition}.



% \paragraph{LRI state for an identity food consumption matrix}
% We would like to know which systems can be in an LRI regime. Consider here the case of $N_R=N_S$ and every species consumes exactly one resource, to each its own. This means the food consumption adjacency matrix is precisely the identity matrix:
% \begin{equation}
% G_{i\mu} = \delta_{i\mu}.
% \end{equation}
% We can then compute explicitly the $\Gamma\Beta$ matrix:
% \begin{equation}
% \left(\Gamma \Beta\right)_{\mu \nu} = \sum_i \left(\alpha_{\mu i}- \gamma_{i\mu}R_\mu^*\right) \sigma_{i\nu}\gamma_{i\nu}S_i^*=\delta_{\mu\nu}\left(\alpha_{\mu\nu}-\gamma_{\nu\mu}R_\mu^*\right)\sigma_{\mu\nu}\gamma_{\mu\nu}S_\nu^*.
% \end{equation}
% Also,
% \begin{equation}
% \sum_\nu \abs{B_{i\nu}} = \sum_\nu \sigma_{i\nu}\gamma_{i\nu} S_i^* = \sigma_{\nu\nu}\gamma_{\nu\nu}S^*_\nu,
% \end{equation}
% and
% \begin{equation}
% \sum_j \abs{\Gamma_{\mu j}}= \abs{\Gamma_{\mu\mu}}=\abs{\alpha_{\mu\mu}-\gamma_{\mu\mu}R^*_\mu}\sigma_{\mu\mu}\gamma_{\mu\mu}S^*_\mu,
% \end{equation}
% such that
% \begin{equation}
% R_C = \max\left(\max_\mu\left\{\sigma_{\mu\mu}\gamma_{\mu\mu}S^*_\mu\right\}, \max_\mu \left\{\abs{\alpha_{\mu\mu}-\gamma_{\mu\mu}R^*_\mu}\sigma_{\mu\mu}\gamma_{\mu\mu}S^*_\mu\right\}\right).
% \end{equation}
% Then we have clearly:
% \begin{empheq}{align}
% (\Gamma\Beta)_{\mu\mu} = \left(\alpha_{\mu\mu}-\gamma_{\mu\mu}R_\mu^*\right)\sigma_{\mu\mu}\gamma_{\mu\mu}S_\mu^*
% \end{empheq}

% \subsection{Flux analysis - a way to get a sense of scales}
% A natural scale free order parameter that at first sight controls the behaviour of the system is the ratio of the syntrophy and consumption fluxes.
%
% The rate of consumption (or \textit{consumption flux}) of species $i$ is given by $\sum_\nu \sigma_{i\nu}\gamma_{i\nu}R_\nu S_i$. Hence the total consumption flux $C_{\text{tot}}$ is given by:
% \begin{equation}
% C_{\text{tot}} = \sum_{i, \nu} \sigma_{i\nu} \gamma_{i\nu}R_\nu S_i.
% \end{equation}
% We can similarly define the total syntrophy flux of the system $S_{\text{tot}}$:
% \begin{equation}
% S_{\text{tot}} = \sum_{i, \nu} \alpha_{\nu i} S_i.
% \end{equation}
% A natural order parameter $O$ is then
% \begin{equation}
% O \equiv \frac{S_{\text{tot}}}{C_{\text{tot}}} = \frac{\sum_{i, \nu} \alpha_{\nu i} S_i}{\sum_{i,\nu}\sigma_{i\nu} \gamma_{i\nu}R_\nu S_i} \approx \frac{N_S N_R \alpha_0 S_0}{\sigma_0 R_0 S_0 N_S N_R \gamma_0} = \frac{\alpha_0}{\sigma_0 R_0 \gamma_0}.
% \end{equation}




\end{document}
