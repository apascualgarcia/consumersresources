\documentclass[12pt, titlepage]{report}
\usepackage{consumer_resource_final}
\graphicspath{{./figures/}}

\begin{document}

We want to be able to build feasible models numerically, \ie we would like to generate a set of constant numbers $\{
l_\nu, m_\nu, R^*_\nu, S^*_j, \gamma_{j\nu}, \alpha_{\nu j}, \sigma_{j\nu}, \tau_{j\nu}\}$ such that the equilibria equations Eqs.\eqref{eq : equilibrium resources and species} are fulfilled.


\subsection{Algorithmic procedure}
We hereby detail the procedure used to numerically build feasible systems. It goes like this:
\begin{enumerate}
  \item We first draw randomly each $R^*_\nu$
  and $S^*_i$ from a uniform distribution of mean equal to the corresponding metaparameter, \ie :
  \begin{equation}
     R^*_\nu = \mathcal{R} \ \forall \nu=1, \dots, N_R\text{ and }  S^*_i = \mathcal{S} \ \forall i=1, \dots, N_S,
  \end{equation}
  where $\mathcal{R}$ and $\mathcal{S}$ are random variables coming from a distribution of mean equal to the corresponding metaparameter and relative standard deviation\footnote{By relative standard deviation, we mean the standard deviation measured in units of the average value.} $\epsilon$. In our simulations, we chose uniform distributions :
  \begin{equation}
  \mathcal{R} \sim \text{Unif}(R_0, \epsilon) \text{ and } \mathcal{S} \sim \text{Unif}(S_0, \epsilon).
  \end{equation}
  \item The efficiency matrix $\sigma_{i\nu}$ is then drawn similarly, from a distribution with average $\sigma_0$. In order to simplify the problem\footnote{Indeed, a non uniform $\sigma_0$ introduces instability in the system}, we will take a zero-variance à la \citeauthor{butler_stability_2018} in \cite{butler_stability_2018}, \ie all species consume resources at the same global efficiency :
  \begin{equation}
    \sigma_{i\nu} = \sigma_0.
  \end{equation}
  \item We build the consumption matrix $\gamma_{i\nu}$. Its adjacency matrix $G$ is loaded through a user-provided file.
  %$G$ is a binary matrix given by the user (in the \code{configuration.in} file) and is defined as the adjacency matrix of the consumption network (\ie it tells which species eats which resource). We then build $\gamma$ with the same network structure as $F$ (\ie both matrices have the same zero elements).
  While $G$ gives the structure of $\gamma$, \ie if a given $\gamma_{i\nu}$ is zero or not, the actual values of $\gamma_{i\nu}$ need then to be determined. They are drawn from a uniform distribution of mean $\gamma_0$ and relative standard deviation $\epsilon$:
  \begin{equation}
    \gamma_{i\nu} = \text{Unif}(\gamma_0, \epsilon) G_{i\nu}.
  \end{equation}
  \item We draw the resources external feeding rates, similarly to the other parameters :
  \begin{equation}
  l_\mu = \text{Unif}(l_0, \epsilon) \ \forall \mu=1, \dots, N_R.
  \end{equation}
  \item The last free parameter is the syntrophy matrix $\alpha_{\nu i}$, the $d_i$ and $l_\mu$ are determined through the equations of evolution at equilibrium. This is the tricky part of the algorithm because $\alpha$ has to follow three constraints, namely energy conservation/dissipation Eq.\eqref{eq : dissipation constraint} and positiveness of $d_i$ and $l_\mu$ [\textbf{insert reference to equation}]. The general strategy is to choose the metaparameters in a way that these constraints should \textit{almost always} be satisfied, \ie we pick metaparameters that follow the feasibility constraint Eq.\eqref{eq : overall feasibility constraint metaparameters}. The adjacency matrix $A$ of $\alpha$ needs then to be specified. At the moment, it can be chosen in three different ways : fully connected, or in a way that no resource eaten by a given species can be released by that same species (\ie $G_{i\mu}>0 \iff A_{\mu i}=0$) or by a user provided matrix. After the adjacency matrix is loaded, we can build $\alpha$ from a uniform distribution of mean $\alpha_0$ and relative standard deviation $\epsilon$:
  \begin{equation}
    \alpha_{\nu i} = \text{Unif}(\alpha_0, \epsilon) A_{\nu i} .
  \end{equation}
  %\item We build $\tau_{\nu i}$. It usually is equal to $\alpha_{\nu i}$ or 0.
  \item With all of these parameters drawn, we can solve Eq.\eqref{eq : equilibrium species} for the species death rate $d_i$.
  \item We solve Eq.\eqref{eq : equilibrium resources} for $m_\nu$. All the parameters of the model are now fully determined.
  \item We check if the constraints Eq.\textbf{(insert reference)} on the parameters are fulfilled. If they are not, we go back to step 1. Otherwise, we can exist the algorithm, a feasible system has been built.
\end{enumerate}

\subsection{The feasibility volume}
The algorithmic procedure above explains how feasible systems can be built. However, it implies that we first found a combination of metaparameters that will most of the time lead to the realisation of feasible systems when they are taken as an input of the algorithm.

Overall we have six metaparameters that we can play with : $\gamma_0$, $\alpha_0$, $l_0$, $\sigma_0$, $S_0$ and $R_0$. However, following the analysis of \cite{barbier_cavity_2017}, we notice that our system \eqref{eq : differential eq for resources and species} is arbitrary on some level. Indeed we have a ``scale freedom'', that means we decide in which set of units we work. There are two physical quantities at stake here : biomass and time, and we may choose, however we want it, a specific set of units describing both of them.

We will measure biomass in units of the average resource abundance at equilibrium\footnote{Note that this is not a completely innocent choice. Indeed we will see later that the matrix $\alpha_{\nu i}-\gamma_{i \nu} R^*_\nu$ is a crucial quantity here. Setting $\av{R^*}=1$ allows us to simply study the impact of $\gamma$ against $\alpha$ instead of the more complicated $\gamma R^*$ versus $\alpha$.}, that means :
\begin{equation}
 \av{R_\mu} = R_0 = 1.
\end{equation}
Similarly, we will measure time such that the average external resource uptake rate is one, that is :
\begin{equation}
\av{l_\mu} = l_0 = 1.
\end{equation}
After this manipulation, our number of metaparameters is reduced from six to four : only $\gamma_0$, $S_0$, $\alpha_0$ and $\sigma_0$ remain.

For the sake of simplicity, we will keep the same $\sigma_0$ throughout our whole study. We take a value close to the efficiency of real microbial systems [\textbf{insert ref}], that is $\sigma_0 =0.3$.

Overall, we need to choose the last three remaining metaparameters: $\alpha_0$, $\gamma_0$ and $S_0$. What Eq.\eqref{eq : overall feasability constraint metaparameters} tells us is that as soon if we choose $\gamma_0$ and $S_0$, we will get a feasability range for $\alpha_0$.
We will then choose $\gamma_0$ and $S_0$ such that they lead to feasible systems for every consumption matrix considered here \textbf{when there is no syntrophy}, \ie $\alpha_0=0$. We will then study the impact of varying $\alpha_0$ at those values of $\gamma_0$ and $S_0$.

Formally, we can define for a consumption adjacency matrix $G$ the volume of the metaparameters space $\mathcal{V}^G_x$ that will lead to a ratio $x$ of feasible systems \ie :
\begin{equation}
V^G_x \defined 
\end{equation}

\end{document}
